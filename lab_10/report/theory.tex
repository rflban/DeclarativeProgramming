\chapter{Теоретические сведения}

\begin{itemize}
    \item Способы организации повторных вычислений в Lisp: функционалы, рекурсия.
    \item Что такое рекурсия? Классификация рекурсивных функций в Lisp: рекурсия — это ссылка на определяемый объект во время его определения; классификация: простая рекурсия - один рекурсивный вызов в теле; рекурсия первого порядка - рекурсивный вызов встречается несколько раз; взаимная рекурсия - используется несколько функций, рекурсивно вызывающих друг друга.
    \item Различные способы организации рекурсивных функций и порядок их реализации: хвостовая (результат формируется не на выходе из рекурсии, а на входе в рекурсию, все действия выполняя до ухода на следующий шаг рекурсии), по нескольким параметрам, дополняемая (при обращении к рекурсивной функции используется дополнительная функция не в аргументе вызова , а вне его), множественная (на одной ветке происходит сразу несколько рекурсивных вызовов).
    \item Способы повышения эффективности реализации рекурсии: применять хвостовую организацию рекурсии.
\end{itemize}

