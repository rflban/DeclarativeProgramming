\chapter{Практическая часть}

\lstset{language=lisp}

\section{Задание \No{}1}
\textbf{Условие:}\\Пусть list-of-lists список, состоящий из списков. Написать функцию, которая вычисляет сумму длин всех элементов list-of-lists.

В литинге 2.1 приведен текст решения данной задачи на языке Common Lisp.

\begin{lstlisting}[caption={Задание \No{}1}]
(defun lvl2len (lol)
  (cond ((null lol) 0)
        (t (+ (length (car lol))
              (lvl2len (cdr lol))))))

(lvl2len nil) ; 0
(lvl2len '((1 2) (3 4))) ; 4
\end{lstlisting}

\section{Задание \No{}2}
\textbf{Условие:}\\Написать рекурсивную версию функции вычисления суммы чисел заданного списка.

В литинге 2.2 приведен текст решения данной задачи на языке Common Lisp.

\begin{lstlisting}[caption={Задание \No{}2}]
(defun reg-add (lst)
  (cond ((null lst) 0)
        (t (+ (cond ((numberp (car lst)) (car lst))
                    (t 0))
              (reg-add (cdr lst))))))

(reg-add '(2 4 6)) ; 12
(reg-add '(2 4 word 6 nil)) ; 12
(reg-add '(word nil)) ; 0
(reg-add nil) ; 0
\end{lstlisting}

\section{Задание \No{}3}
\textbf{Условие:}\\Написать рекурсивную версию функции nth.

В литинге 2.3 приведен текст решения данной задачи на языке Common Lisp.

\begin{lstlisting}[caption={Задание \No{}3}]
(defun recnth (idx lst)
  (cond ((or (= 0 idx)
             (null lst)) (car lst))
        (t (recnth (- idx 1) (cdr lst)))))

(recnth 0 '(1 2 3)) ; 1
(recnth 2 '(1 2 3)) ; 3
(recnth 1 '(1 2 3)) ; 2
(recnth 5 '(1 2 3)) ; NIL
\end{lstlisting}

\section{Задание \No{}4}
\textbf{Условие:}\\Написать рекурсивную функцию, которая возвращает t когда все элементы списка нечетные.

В литинге 2.4 приведен текст решения данной задачи на языке Common Lisp.

\begin{lstlisting}[caption={Задание \No{}4}]
(defun alloddr (lst)
  (cond ((null lst) t)
        ((evenp (car lst)) nil)
        (t (alloddr (cdr lst)))))

(alloddr '(1 2 3)) ; nil
(alloddr '(1 3 5)) ; nil
\end{lstlisting}

\section{Задание \No{}5}
\textbf{Условие:}\\Написать рекурсивную функцию, относящуюся к хвостовой рекурсии с одним тестом завершения, которая возвращает последний элемент списка-аргумента.

В литинге 2.5 приведен текст решения данной задачи на языке Common Lisp.

\begin{lstlisting}[caption={Задание \No{}5}]
(defun latest (lst)
  (cond ((null (cdr lst)) (car lst))
        (t (latest (cdr lst)))))

(latest nil) ; NIL
(latest '(1)) ; 1
(latest '(1 2 3 (4 5))) ; (4 5)
\end{lstlisting}

\section{Задание \No{}6}
\textbf{Условие:}\\Написать рекурсивную функцию, относящуюся к дополняемой рекурсии с одним тестом завершения, которая вычисляет сумму всех чисел:
\begin{enumerate}
    \item от 0 до n-ого аргумента функции;
    \item от n-аргумента функции до последнего >= 0;
    \item от n-аргумента функции до r-аргумента с шагом d.
\end{enumerate}

В литинге 2.6 приведен текст решения данной задачи на языке Common Lisp.

\begin{lstlisting}[caption={Задание \No{}6}]
;;; 1)
(defun sum-first (nums n)
  (cond ((or (= 0 n)
             (null nums)) 0)
        (t (+ (car nums)
              (sum-first (cdr nums) (- n 1))))))

;;; 2)
(defun help1 (nums n)
  (cond ((or (null nums) (= 0 n)) 0)
        (t (+ (car nums)
              (help1 (cdr nums) (- n 1))))))

(defun help2 (nums n)
  (cond ((or (null nums) (= 0 n)) 0)
        ((> (car nums) 0) (help1 nums n))
        (t (help2 (cdr nums) (- n 1)))))

(defun sum-to-last-pos (nums n)
  (help2 (reverse nums) (- (length nums) n)))

;;; 3)
(defun sum-range (nums lo hi h)
  (cond ((or (= 0 hi)
             (null nums)) 0)
        (t (+ (sum-range (cdr nums)
                         (cond ((= lo 0) h)
                               (t (- lo 1)))
                         (- hi 1)
                         h)
              (cond ((> lo 0) 0)
                    (t (car nums)))))))

(setq nums '(0 1 2 3 4 5 6 7 8 9))

(sum-first nums 5) ; 10
(sum-first nums 0) ; 0
(sum-first nums 1) ; 0
(sum-first nums 2) ; 1
(sum-first nums 10) ; 45

(sum-to-last-pos nums 5) ; 35
(sum-to-last-pos '(-1 2 3 4 -1) 2) ; 7
(sum-to-last-pos (concatenate 'list nums '(-10 5 -7 -9 -2)) 5) ; 30

(sum-range nums 1 9 2) ; 12
(sum-range nums 0 10 0) ; 45
(sum-range nums 0 10 1) ; 20
(sum-range nums 1 10 1) ; 25
\end{lstlisting}

\section{Задание \No{}7}
\textbf{Условие:}\\Написать рекурсивную функцию, которая возвращает последнее нечетное число из числового списка.

В литинге 2.7 приведен текст решения данной задачи на языке Common Lisp.

\begin{lstlisting}[caption={Задание \No{}7}]
(defun help1 (lst lodd)
  (cond ((null lst) lodd)
        (t (help1 (cdr lst)
                  (cond ((oddp (car lst)) (car lst))
                        (t lodd))))))

(defun lastest-odd (lst)
  (help1 lst nil))

(lastest-odd nil) ; NIL
(lastest-odd '(2 4 6 8)) ; NIL
(lastest-odd '(1 2 3 4 5 6 7 8 9 -1 10)) ; -1
\end{lstlisting}

\section{Задание \No{}8}
\textbf{Условие:}\\Используя cons-дополняемую рекурсию с одним тестом завершения, написать функцию которая получает как аргумент список чисел, а возвращает список квадратов этих чисел в том же порядке.

В литинге 2.8 приведен текст решения данной задачи на языке Common Lisp.

\begin{lstlisting}[caption={Задание \No{}8}]
(defun sqr-nums (nums)
  (cond ((null nums) nil)
        (t (cons (* (car nums) (car nums))
                 (sqr-nums (cdr nums))))))

(sqr-nums nil) ; NIL
(sqr-nums '(1 2 -3 4 5)) ; (1 4 9 16 25)
\end{lstlisting}

\section{Задание \No{}9}
\textbf{Условие:}\\Написать функцию, которая:
\begin{enumerate}
    \item выбирает из списка чисел все чётные;
    \item выбирает из списка чисел все нечётные;
    \item суммирует все четные числа списка;
    \item суммирует все нечетные числа списка.
\end{enumerate}

В литинге 2.9 приведен текст решения данной задачи на языке Common Lisp.

\begin{lstlisting}[caption={Задание \No{}9}]
(defun filter (predicate lst)
  (cond ((null lst) nil)
        (t (cond ((funcall predicate (car lst))
                  (cons (car lst) (filter predicate (cdr lst))))
                 (t (filter predicate (cdr lst)))))))

(defun select-odd (nums)
  (filter #'oddp nums))

(defun select-even (nums)
  (filter #'evenp nums))

(defun sum-all-odd (nums)
  (reduce #'+ (select-odd nums)))

(defun sum-all-even (nums)
  (reduce #'+ (select-even nums)))

(setq digits '(0 1 2 3 4 5 6 7 8 9))

(select-odd digits) ; (1 3 5 7 9)
(sum-all-odd digits) ; 25

(select-even digits) ; (0 2 4 6 8)
(sum-all-even digits) ; 20
\end{lstlisting}

