\chapter{Теоретические сведения}

\begin{itemize}
    \item Способы организации повторных вычислений в Lisp: функционалы, рекурсия.
    \item Различные способы использования функционалов: функционалы-предикаты; функционалы, использующие предикаты в качестве функционального объекта.
    \item Что такое рекурсия? Способы организации рекурсивных функци: рекурсия — это ссылка на определяемый объект во время его определения; способы организации: хвостовая, по нескольким параметрам, дополняемая, множественная.
    \item Способы повышения эффективности реализации рекурсии: применять хвостовую организацию рекурсии.
\end{itemize}

