\chapter{Практическая часть}

\lstset{language=lisp}

\section{Задание \No{}1}
\textbf{Условие:}\\Написать предикат, который возвращает t, если два его множество-аргумента содержат одни и те же элементы, порядок которых не имеет значения.

В литинге 2.1 приведен текст решения дааной задачи на языке Common Lisp.

\begin{lstlisting}[caption={Задание \No{}1}]
;;; fun
(defun set-equal (l1 l2)
  (and (= (length l1) (length l2))
       (reduce #'(lambda (val1 val2) (and val1 val2))
               (mapcar #'(lambda (e1)
                           (reduce #'(lambda (val1 val2) (or val1 val2))
                                   (mapcar #'(lambda (e2)
                                               (equal e1 e2))
                                           l2)))
                       l1))))

;;; rec
(defun help1 (e1 l2)
  (cond ((null l2) nil)
        (t (or (equal e1 (car l2))
               (help1 e1 (cdr l2))))))

(defun help2 (l1 l2)
  (cond ((null l1) t)
        (t (and (help1 (car l1) l2)
                (help2 (cdr l1) l2)))))

(defun set-equal (l1 l2)
  (and (= (length l1) (length l2))
       (help2 l1 l2)))

;;; test
(set-equal '(1 2 3) '(1 3 2)) ; T
(set-equal '(1 2 3) '(1 5 2 3)) ; NIL
(set-equal '(1 5 2 3) '(1 2 3)) ; NIL
\end{lstlisting}

Способ с использованием функционалов состоит в том, что необходимо составить список-маску, в котором каждый элемент соотвествует по порядку каждому элементу из первого списка-аргумента функции. Если во втором списке есть хотя бы один элемент, равный $i$-му элементу первого списка, то $i$-ый элемент маски равен \textbf{t}, иначе - \textbf{nil}. Если в списке-маске есть хотя бы один \textbf{nil}, то весь результат равен \textbf{nil}, иначе -  \textbf{t}.

Рекурсивный способ решения данной задачи так же заключается в полном переборе: каждый элемент из первого списка сравнивается с каждым из второго, если нет ни одного совпадения, то ответ - \textbf{nil}, иначе - \textbf{t}.

В обоих случаях присутствует начальная проверка на совпадение длин списков-множеств, которая для равных должна совпадать.

\section{Задание \No{}2}
\textbf{Условие:}\\Напишите необходимые функции, которые обрабатывают таблицу из точечных пар: (страна . столица), и возвращают по стране - столицу, а по столице - страну.

В литинге 2.2 приведен текст решения дааной задачи на языке Common Lisp.

\begin{lstlisting}[caption={Задание \No{}2}]
;;; fun
(defun table_create (countries capitals)
  (mapcar #'cons countries capitals))

(defun get_capital (tbl country)
  (reduce #'(lambda (val1 val2) (or val1 val2))
          (mapcar #'(lambda (row)
                      (cond ((equal (car row) country) (cdr row))
                            (t nil)))
                  tbl)))

(defun get_country (tbl capital)
  (reduce #'(lambda (val1 val2) (or val1 val2))
          (mapcar #'(lambda (row)
                      (cond ((equal (cdr row) capital) (car row))
                            (t nil)))
                  tbl)))

;;; rec
(defun table_create (countries capitals)
  (cond ((cdr countries)
         (cons (cons (car countries) (car capitals))
               (table_create (cdr countries) (cdr capitals))))
        (t
         (list (cons (car countries) (car capitals))))))

(defun get_capital (tbl country)
  (cond ((null tbl) nil)
        ((equal (caar tbl) country) (cdar tbl))
        (t (get_capital (cdr tbl) country))))

(defun get_country (tbl capital)
  (cond ((null tbl) nil)
        ((equal (cdar tbl) capital) (caar tbl))
        (t (get_country (cdr tbl) capital))))

;;; test
(setq countries '(Russia France Germany))
(setq capitals '(Moscow Paris Berlin))
(setq table (table_create countries capitals))

(get_capital table 'Russia) ; MOSCOW
(get_capital table 'Germany) ; BERLIN
(get_capital table 'France) ; PARIS
(get_capital table 'USA) ; NIL

(get_country table 'Berlin) ; GERMANY
(get_country table 'Moscow) ; RUSSIA
(get_country table 'Paris) ; FRANCE
(get_country table 'Washington) ; NIL
\end{lstlisting}

\section{Задание \No{}3}
\textbf{Условие:}\\Напишите функцию, которая умножает на заданное число-аргумент все числа из заданного списка-аргумента, когда:
\begin{enumerate}
	\item все элементы списка - числа;
	\item элементы списка -- любые объекты.
\end{enumerate}

В литинге 2.3 приведен текст решения дааной задачи на языке Common Lisp.

\begin{lstlisting}[caption={Задание \No{}3}]
;;; fun
(defun mul1 (lst num)
  (mapcar #'(lambda (item)
              (* num item))
          lst))

(defun mul2 (lst num)
  (mapcar #'(lambda (item)
              (cond ((numberp item) (* num item))
                    (t item)))
          lst))

;;; rec
(defun mul1 (lst num)
  (cond ((null lst) nil)
        (t (cons (* (car lst) num) (mul1 (cdr lst) num)))))

(defun mul2 (lst num)
  (cond ((null lst) nil)
        (t (cons (cond ((numberp (car lst)) (* (car lst) num))
                       (t (car lst)))
                 (mul2 (cdr lst) num)))))

;;; test
(setq nums '(4 21 5 43 31 8))
(setq vals '(4 nil 5 43 (5 7) 8))

(identity nums) ; (4 21 5 43 31 8)
(mul1 nums 4) ; (16 84 20 172 124 32)

(identity vals) ; (4 NIL 5 43 (5 7)
(mul2 vals 4) ; (16 NIL 20 172 (5 7)
\end{lstlisting}

\section{Задание \No{}4}
\textbf{Условие:}\\Напишите функцию, которая уменьшает на 10 все числа из списка-аргумента этой функции.

В литинге 2.4 приведен текст решения дааной задачи на языке Common Lisp.

\begin{lstlisting}[caption={Задание \No{}4}]
;;; fun
(defun sub10 (lst)
  (mapcar #'(lambda (val)
              (cond ((numberp val) (- val 10))
                    (t val)))
          lst))

;;; rec
(defun sub10 (lst)
  (cond ((null lst) nil)
        (t (cons (cond ((numberp (car lst)) (- (car lst) 10))
                       (t (car lst)))
                 (sub10 (cdr lst))))))

;;; test
(setq lst1 '(-2 4 1 2 4 nil 5 8 12 (555 666) 9 -99 10 7))
(sub10 lst1) ; (-12 -6 -9 -8 -6 NIL -5 -2 2 (555 666) -1 -109 0 -3)
\end{lstlisting}

\section{Задание \No{}5}
\textbf{Условие:}\\Написать функцию, которая возвращает первый аргумент списка-аргумента, который сам является непустым списком.

В литинге 2.5 приведен текст решения дааной задачи на языке Common Lisp.

\begin{lstlisting}[caption={Задание \No{}5}]
;;; fun
(defun find-no-empty (lst)
  (reduce #'(lambda (val1 val2) (or val1 val2))
          (mapcar #'(lambda (item)
                      (cond ((listp item) item)
                            (t nil)))
                  lst)))

;;; rec
(defun find-no-empty (lst)
  (cond ((null lst) nil)
        ((and (not (null (car lst)))
              (listp (car lst))) (car lst))
        (t (find-no-empty (cdr lst)))))

;;; test
(find-no-empty '(1 23 nil -6 Word (1 2 3) () 66 (4 5 6))) ; (1 2 3)
(find-no-empty '((3 4))) ; (3 4)
(find-no-empty '(1 2)) ; nil
(find-no-empty  '()) ; nil
\end{lstlisting}

\section{Задание \No{}6}
\textbf{Условие:}\\Написать функцию, которая выбирает из заданного списка только те числа, которые больше 1 и меньше 10.

В литинге 2.6 приведен текст решения дааной задачи на языке Common Lisp.

\begin{lstlisting}[caption={Задание \No{}6}]
;;; fun
(defun filter (predicate lst)
  (reverse (reduce #'(lambda (state item)
                       (cond ((funcall predicate item) (cons item state))
                             (t state)))
                   (cons nil lst))))

;;; rec
(defun filter (predicate lst)
  (cond ((null lst) nil)
        (t (cond ((funcall predicate (car lst))
                  (cons (car lst) (filter predicate (cdr lst))))
                 (t (filter predicate (cdr lst)))))))

;;; common
(defun interval-filter (lst lo hi)
  (filter #'(lambda (item)
              (cond ((numberp item) (or (and (> item lo) (< item hi))
                                        (and (> item hi) (< item lo))))
                    (t nil)))
          lst))

;;; test
(setq lst1 '(-2 4 1 2 4 nil 5 8 12 (555 666) 9 -99 10 7))

(interval-filter lst1 1 10) ; (4 2 4 5 8 9 7)
(interval-filter lst1 10 1) ; (4 2 4 5 8 9 7)
(interval-filter lst1 100 1000) ; NIL
(interval-filter lst1 1 1) ; NIL
\end{lstlisting}

\section{Задание \No{}7}
\textbf{Условие:}\\Написать функцию, вычисляющую декартово произведение двух своих списков-аргументов.

В литинге 2.7 приведен текст решения дааной задачи на языке Common Lisp.

\begin{lstlisting}[caption={Задание \No{}7}]
;;; fun
(defun cart-product (l1 l2)
  (mapcan #'(lambda (e1)
              (mapcar #'(lambda (e2) (list e1 e2))
                      l2))
          l1))

;;; rec
(defun help1 (e1 l2)
  (cond ((null l2) nil)
        (t (cons (list e1 (car l2))
                 (help1 e1 (cdr l2))))))

(defun cart-product (l1 l2)
  (cond ((null l1) nil)
        (t (nconc (help1 (car l1) l2)
                  (cart-product (cdr l1) l2)))))

;;; test
(cart-product nil '(3 4)) ; NIL
(cart-product '(1 2) nil) ; NIL
(cart-product '(buy sell) '(car cycle))
; ((BUY CAR) (BUY CYCLE) (SELL CAR) (SELL CYCLE))
\end{lstlisting}

\section{Задание \No{}8}
\textbf{Условие:}\\Почему так реализовано reduce, в чем причина?\\(reduce #*+0) ->\\(reduce #*+ ()) -> 0

Этот текст не компилируется. Возможно, вместо него подразумевался текст, приведенный в листинге 2.8.
\begin{lstlisting}[caption={Задание \No{}8}]
(reduce #'+ nil) ; 0
(reduce #'* nil) ; 1
\end{lstlisting}

В данном случае, результат первой строчки совпадает с результатом вычисления \textbf{(+)}, а второй - с результатом вычисления \textbf{(*)}, что очевидно, так как список аргументов в обоих случаях пуст (равен \textbf{nil}).

Такое поведение функции reduce приведёт к ошибке при использовании в качестве функционального объекта функции, не вычисляющейся при пустом списке аргументов.

Тем не менее, функция reduce имеет особенность: если используемый список аргументов состоит только из одного элемента, то в результате вычисления функции reduce будет получено значение именно этого элемента. Данный нюанс можно объяснить тем, что от функции reduce ожидается, что при использовании функции от двух аргументов в качестве функционального объекта и списка параметров, содержащего один или более элемент, она должна вычисляться без ошибок времени выполнения.

