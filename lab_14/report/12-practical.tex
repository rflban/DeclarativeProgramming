\chapter{Практическая часть}
В этом разделе приведено описание задания и его решения.

\section{Условие}
Составить программу, т.е. модель предметной области~--- базу знаний, объединив в ней информацию~--- знания:
\begin{itemize}
    \item \textbf{<<Телефонный справочник>>}: Фамилия, \textnumero{}тел, Адрес~--- структура (Город, Улица, \textnumero{}дома, \textnumero{}кв);
    \item \textbf{<<Автомобили>>}:  Фамилия\_владельца, Марка, Цвет, Стоимость, и др.;
    \item \textbf{<<Вкладчики банков>>}: Фамилия, Банк, счет, сумма, др.
\end{itemize}

Владелец может иметь несколько телефонов, автомобилей, вкладов (Факты). В разных городах есть однофамильцы, в одном городе – фамилия уникальна.

Используя \textbf{конъюнктивное правило и простой вопрос}, обеспечить возможность поиска:

По Марке и Цвету автомобиля найти Фамилию, Город, Телефон и Банки, в которых владелец автомобиля имеет вклады. Лишней информации не находить и не передавать!!!

Владельцев может быть \textbf{несколько} (не более 3-х), \textbf{один и ни одного}.
\begin{enumerate}
    \item Для каждого из трех вариантов словесно подробно описать порядок формирования ответа (в виде таблицы). При этом, указать~--– отметить моменты очередного запуска алгоритма унификации и полный результат его работы. Обосновать следующий шаг работы системы. Выписать унификаторы~--– подстановки. Указать моменты, причины и результат отката, если он есть.

    \item Для случая нескольких владельцев (2-х):\newline{}приведите примеры (таблицы) работы системы при разных порядках следования в БЗ процедур, и знаний в них: (<<Телефонный справочник>>, <<Автомобили>>, <<Вкладчики банков>>, или: <<Автомобили>>, <<Вкладчики банков>>, <<Телефонный справочник>>). Сделайте вывод: Одинаковы ли: множество работ и объем работ в разных случаях?

    \item Оформите 2 таблицы, демонстрирующие порядок работы алгоритма унификации вопроса и подходящего заголовка правила (для двух случаев из пункта 2) и укажите результаты его работы: ответ и побочный эффект.
\end{enumerate}

\section{База знаний}

В листинге~\ref{lst:knowledge} приведён текст базы знаний, удовлетворяющей условию задачи.
\begin{lstlisting}[caption={База знаний},label={lst:knowledge}]
domains
  lastname, city, street, carbrand, color, bankname = symbol.
  phonenum, housenum, apartnum, price, account, deposit = integer.
  address = address(city, street, housenum, apartnum).
predicates
  phonebook(lastname, phonenum, address).
  car(lastname, carbrand, color, price).
  depositor(lastname, bankname, account, deposit).

  search(carbrand, color, lastname, city, phonenum, bankname).
clauses
  phonebook(petrov, 74400297,
            address(moscow,    lenina,       4,  2)).
  phonebook(igorev, 77270935,
            address(moscow,    marksa,       3,  5)).
  phonebook(ivanov, 79345669,
            address(moscow,    pushkinskaya, 11, 1)).
  phonebook(stasov, 74024456,
            address(spb,       marksa,       4,  4)).
  phonebook(alkema, 73148253,
            address(ekb,       marksa,       6,  8)).
  phonebook(igorev, 73243243,
            address(volgograd, lenina,       9,  9)).

  car(petrov, ford,    pink,   24000).
  car(petrov, ferrari, red,    55000).
  car(ivanov, ford,    pink,   25000).
  car(igorev, tesla,   purple, 44000).
  car(stasov, bmw,     green,  3700).
  car(alkema, lexus,   yellow, 14000).

  depositor(petrov, agricole, 5, 52150322).
  depositor(igorev, paribas,  4, 32242424).
  depositor(ivanov, sberbank, 6, 442423123).
  depositor(stasov, sberbank, 1, 423424233).
  depositor(igorev, paribas,  3, 41424214).
  depositor(igorev, sberbank, 8, 421342352).

  search(Carbrand, Color, Lastname, City, Phonenum, Bankname) :-
    car(Lastname, Carbrand, Color, _),
    phonebook(Lastname, Phonenum, address(City, _, _, _)),
    depositor(Lastname, Bankname, _, _).
\end{lstlisting}

\section{Задание \textnumero1}
Рассмотрим примеры трёх целей, текст и решение которых приведены в листингах~\ref{lst:ex1},~\ref{lst:ex2},~\ref{lst:ex3}.

\begin{lstlisting}[caption={Пример \textnumero1},label={lst:ex1}]
goal
  search(ford, pink, Lastname, City, Phonenum, Bankname).

% Lastname=petrov, City=moscow, Phonenum=74400297, Bankname=agricole
% Lastname=ivanov, City=moscow, Phonenum=79345669, Bankname=sberbank
% 2 Solutions
\end{lstlisting}

\begin{lstlisting}[caption={Пример \textnumero2},label={lst:ex2}]
goal
  search(bmw, green, Lastname, City, Phonenum, Bankname).

% Lastname=stasov, City=spb, Phonenum=74024456, Bankname=sberbank
% 1 Solution
\end{lstlisting}

\begin{lstlisting}[caption={Пример \textnumero3},label={lst:ex3}]
goal
  search(lexus, yellow, Lastname, City, Phonenum, Bankname).

% No Solutions
\end{lstlisting}

В таблицах~\ref{tbl:ex1},~\ref{tbl:ex2},~\ref{tbl:ex3} приведены подробные описания поиска решений целей из листингов~\ref{lst:ex1},~\ref{lst:ex2},~\ref{lst:ex3} соответственно.

\renewcommand{\arraystretch}{1.75}
\footnotesize
\begin{longtable}{|c|p{.4625\textwidth}|p{.3625\textwidth}|}
    \caption{Поиск решения в примере \textnumero1}\label{tbl:ex1} \\
    \hline
    \textnumero{} шага & Сравниваемые термы; результат; подстановка (если есть)  & Дальнейшие действия: прямой ход или откат \\
    \hline

    1
                       &
\textbf{Сравнение:} \newline
    search(ford, pink, Lastname, City, Phonenum, Bankname)
    \newline и \newline
    search(Carbrand, Color, Lastname, City, Phonenum, Bankname).
    \newline

    \textbf{Результат:} \newline
    Унификация выполнена.\newline{}Carbrand=ford; Color=pink.
                       &
                       В стек откладывается\newline
                       search(ford, pink, Lastname, City, Phonenum, Bankname).
                       \newline

                       \textbf{Прямой ход.}

                       На резольвенту попадают:

                       car(Lastname, Carbrand, Color, \_),

                       phonebook(Lastname, Phonenum, address(City, \_, \_, \_)),

                       depositor(Lastname, Bankname, \_, \_).
                       \\ \hline

    2
                       &
\textbf{Сравнение:} \newline
    car(petrov, ford,    pink,   24000)
    \newline и \newline
    car(Lastname, ford, pink, \_).
    \newline

    \textbf{Результат:} \newline
    Унификация выполнена.\newline{}
    Lastname=petrov.
                       &
    В стек откладывается\newline
    car(Lastname, ford, pink, \_).
    \newline

    \textbf{Прямой ход.}
                       \\ \hline

    3
                       &
\textbf{Сравнение:} \newline
    phonebook(petrov, 74400297, address(moscow,    lenina,       4,  2))
    \newline и \newline
    phonebook(petrov, Phonenum, address(City, \_, \_, \_)).
    \newline

    \textbf{Результат:} \newline
    Унификация выполнена.\newline{}
    Phonenum=74400297, City=moscow.
                       &
    В стек откладывается\newline
    phonebook(petrov, Phonenum, address(City, \_, \_, \_)).
    \newline

    \textbf{Прямой ход.}
                       \\ \hline

    4
                       &
\textbf{Сравнение:} \newline
    depositor(petrov, agricole, 5, 52150322)
    \newline и \newline
    depositor(petrov, Bankname, \_).
    \newline

    \textbf{Результат:} \newline
    Унификация выполнена.\newline{}
    Bankname=agricole.
                       &
    \textbf{Получено первое решение:}\newline
    Lastname=petrov, City=moscow, Phonenum=74400297, Bankname=agricole.
    \newline

    В стек откладывается\newline
    depositor(petrov, Bankname, \_).
    \newline

    \textbf{Прямой ход.}
                       \\ \hline

    5
                       &
    Резольвента пуста.
    \newline

    \textbf{Результат:} \newline
    Тупик, унификация провалена.
                       &
    \textbf{Откат.}
    \newline

    Из стека восстанавливается терм \newline
    depositor(petrov, Bankname, \_).
                       \\ \hline

    6
                       &
\textbf{Сравнение:} \newline
    depositor(igorev, paribas,  4, 32242424)
    \newline и \newline
    depositor(petrov, Bankname, \_).
    \newline

    \textbf{Результат:} \newline
    Унификация не выполнена.
                       &
    \textbf{Прямой ход.}
                       \\ \hline

    7
                       &
\textbf{Сравнение:} \newline
    depositor(ivanov, sberbank, 6, 442423123)
    \newline и \newline
    depositor(petrov, Bankname, \_).
    \newline

    \textbf{Результат:} \newline
    Унификация не выполнена.
                       &
    \textbf{Прямой ход.}
                       \\ \hline

    8
                       &
\textbf{Сравнение:} \newline
    depositor(stasov, sberbank, 1, 423424233)
    \newline и \newline
    depositor(petrov, Bankname, \_).
    \newline

    \textbf{Результат:} \newline
    Унификация не выполнена.
                       &
    \textbf{Прямой ход.}
                       \\ \hline

    9
                       &
\textbf{Сравнение:} \newline
    depositor(igorev, paribas,  3, 41424214)
    \newline и \newline
    depositor(petrov, Bankname, \_).
    \newline

    \textbf{Результат:} \newline
    Унификация не выполнена.
                       &
    \textbf{Прямой ход.}
                       \\ \hline

    10
                       &
\textbf{Сравнение:} \newline
    depositor(igorev, sberbank, 8, 421342352)
    \newline и \newline
    depositor(petrov, Bankname, \_).
    \newline

    \textbf{Результат:} \newline
    Унификация не выполнена.
                       &
    \textbf{Прямой ход.}
                       \\ \hline

    11
                       &
    Процедура исчерпана.
    \newline

    \textbf{Результат:} \newline
    Тупик, унификация провалена.
                       &
    \textbf{Откат.}
    \newline

    Из стека восстанавливается терм \newline
    phonebook(petrov, Phonenum, address(City, \_, \_, \_)).
                       \\ \hline

    12
                       &
\textbf{Сравнение:} \newline
    phonebook(igorev, 77270935,
              address(moscow,    marksa,       3,  5)).
    \newline и \newline
    phonebook(petrov, Phonenum, address(City, \_, \_, \_)).
    \newline

    \textbf{Результат:} \newline
    Унификация не выполнена.
                       &
    \textbf{Прямой ход.}
                       \\ \hline

    13
                       &
\textbf{Сравнение:} \newline
    phonebook(ivanov, 79345669,
              address(moscow,    pushkinskaya, 11, 1)).
    \newline и \newline
    phonebook(petrov, Phonenum, address(City, \_, \_, \_)).
    \newline

    \textbf{Результат:} \newline
    Унификация не выполнена.
                       &
    \textbf{Прямой ход.}
                       \\ \hline

    14
                       &
\textbf{Сравнение:} \newline
    phonebook(stasov, 74024456,
              address(spb,       marksa,       4,  4)).
    \newline и \newline
    phonebook(petrov, Phonenum, address(City, \_, \_, \_)).
    \newline

    \textbf{Результат:} \newline
    Унификация не выполнена.
                       &
    \textbf{Прямой ход.}
                       \\ \hline

    15
                       &
\textbf{Сравнение:} \newline
    phonebook(alkema, 73148253,
              address(ekb,       marksa,       6,  8)).
    \newline и \newline
    phonebook(petrov, Phonenum, address(City, \_, \_, \_)).
    \newline

    \textbf{Результат:} \newline
    Унификация не выполнена.
                       &
    \textbf{Прямой ход.}
                       \\ \hline

    16
                       &
\textbf{Сравнение:} \newline
    phonebook(igorev, 73243243,
              address(volgograd, lenina,       9,  9)).
    \newline и \newline
    phonebook(petrov, Phonenum, address(City, \_, \_, \_)).
    \newline

    \textbf{Результат:} \newline
    Унификация не выполнена.
                       &
    \textbf{Прямой ход.}
                       \\ \hline

    17
                       &
    Процедура исчерпана.
    \newline

    \textbf{Результат:} \newline
    Тупик, унификация провалена.
                       &
    \textbf{Откат.}
    \newline

    Из стека восстанавливается терм \newline
    car(Lastname, ford, pink, \_).
                       \\ \hline

    18
                       &
\textbf{Сравнение:} \newline
    car(petrov, ferrari, red,    55000)
    \newline и \newline
    car(Lastname, ford, pink, \_).
    \newline

    \textbf{Результат:} \newline
    Унификация не выполнена.
                       &
    \textbf{Прямой ход.}
                       \\ \hline

    19
                       &
\textbf{Сравнение:} \newline
    car(ivanov, ford,    pink,   25000)
    \newline и \newline
    car(Lastname, ford, pink, \_).
    \newline

    \textbf{Результат:} \newline
    Унификация выполнена.\newline{}
    Lastname=ivanov.
                       &
    В стек откладывается\newline
    car(Lastname, ford, pink, \_).
    \newline

    \textbf{Прямой ход.}
                       \\ \hline

    20
                       &
\textbf{Сравнение:} \newline
    phonebook(petrov, 74400297, address(moscow,    lenina,       4,  2))
    \newline и \newline
    phonebook(ivanov, Phonenum, address(City, \_, \_, \_)).
    \newline

    \textbf{Результат:} \newline
    Унификация не выполнена.
                       &
    \textbf{Прямой ход.}
                       \\ \hline

    21
                       &
\textbf{Сравнение:} \newline
  phonebook(igorev, 77270935,
            address(moscow,    marksa,       3,  5)).
    \newline и \newline
    phonebook(ivanov, Phonenum, address(City, \_, \_, \_)).
    \newline

    \textbf{Результат:} \newline
    Унификация не выполнена.
                       &
    \textbf{Прямой ход.}
                       \\ \hline

    22
                       &
\textbf{Сравнение:} \newline
  phonebook(ivanov, 79345669,
            address(moscow,    pushkinskaya, 11, 1)).
    \newline и \newline
    phonebook(ivanov, Phonenum, address(City, \_, \_, \_)).
    \newline

    \textbf{Результат:} \newline
    Унификация выполнена.\newline{}
    Phonenum=79345669, City=moscow.
                       &
    В стек откладывается\newline
    phonebook(petrov, Phonenum, address(City, \_, \_, \_)).
    \newline

    \textbf{Прямой ход.}
                       \\ \hline

    23
                       &
\textbf{Сравнение:} \newline
  depositor(petrov, agricole, 5, 52150322)
    \newline и \newline
    depositor(ivanov, Bankname, \_).
    \newline

    \textbf{Результат:} \newline
    Унификация не выполнена.
                       &
    \textbf{Прямой ход.}
                       \\ \hline

    24
                       &
\textbf{Сравнение:} \newline
  depositor(igorev, paribas,  4, 32242424)
    \newline и \newline
    depositor(ivanov, Bankname, \_).
    \newline

    \textbf{Результат:} \newline
    Унификация не выполнена.
                       &
    \textbf{Прямой ход.}
                       \\ \hline

    25
                       &
\textbf{Сравнение:} \newline
  depositor(ivanov, sberbank, 6, 442423123)
    \newline и \newline
    depositor(ivanov, Bankname, \_).
    \newline

    \textbf{Результат:} \newline
    Унификация выполнена.\newline{}
    Bankname=sberbank.
                       &
    \textbf{Получено второе решение:}\newline
    Lastname=ivanov, City=moscow, Phonenum=79345669, Bankname=sberbank.
    \newline

    В стек откладывается\newline
    depositor(ivanov, Bankname, \_).
    \newline

    \textbf{Прямой ход.}
                       \\ \hline

    26
                       &
    Резольвента пуста.
    \newline

    \textbf{Результат:} \newline
    Тупик, унификация провалена.
                       &
    \textbf{Откат.}
    \newline

    Из стека восстанавливается терм \newline
    depositor(ivanov, Bankname, \_).
                       \\ \hline

    27
                       &
\textbf{Сравнение:} \newline
  depositor(stasov, sberbank, 1, 423424233)
    \newline и \newline
    depositor(ivanov, Bankname, \_).
    \newline

    \textbf{Результат:} \newline
    Унификация не выполнена.
                       &
    \textbf{Прямой ход.}
                       \\ \hline

    28
                       &
\textbf{Сравнение:} \newline
  depositor(igorev, paribas,  3, 41424214)
    \newline и \newline
    depositor(ivanov, Bankname, \_).
    \newline

    \textbf{Результат:} \newline
    Унификация не выполнена.
                       &
    \textbf{Прямой ход.}
                       \\ \hline

    28
                       &
\textbf{Сравнение:} \newline
  depositor(igorev, sberbank, 8, 421342352)
    \newline и \newline
    depositor(ivanov, Bankname, \_).
    \newline

    \textbf{Результат:} \newline
    Унификация не выполнена.
                       &
    \textbf{Прямой ход.}
                       \\ \hline

    29
                       &
    Процедура исчерпана.
    \newline

    \textbf{Результат:} \newline
    Тупик, унификация провалена.
                       &
    \textbf{Откат.}
    \newline

    Из стека восстанавливается терм \newline
    phonebook(ivanov, Phonenum, address(City, \_, \_, \_)).
                       \\ \hline

    30
                       &
\textbf{Сравнение:} \newline
  phonebook(stasov, 74024456,
            address(spb,       marksa,       4,  4))
    \newline и \newline
    phonebook(ivanov, Phonenum, address(City, \_, \_, \_)).
    \newline

    \textbf{Результат:} \newline
    Унификация не выполнена.
                       &
    \textbf{Прямой ход.}
                       \\ \hline

    31
                       &
\textbf{Сравнение:} \newline
  phonebook(alkema, 73148253,
            address(ekb,       marksa,       6,  8))
    \newline и \newline
    phonebook(ivanov, Phonenum, address(City, \_, \_, \_)).
    \newline

    \textbf{Результат:} \newline
    Унификация не выполнена.
                       &
    \textbf{Прямой ход.}
                       \\ \hline

    32
                       &
\textbf{Сравнение:} \newline
  phonebook(igorev, 73243243,
            address(volgograd, lenina,       9,  9))
    \newline и \newline
    phonebook(ivanov, Phonenum, address(City, \_, \_, \_)).
    \newline

    \textbf{Результат:} \newline
    Унификация не выполнена.
                       &
    \textbf{Прямой ход.}
                       \\ \hline

    33
                       &
    Процедура исчерпана.
    \newline

    \textbf{Результат:} \newline
    Тупик, унификация провалена.
                       &
    \textbf{Откат.}
    \newline

    Из стека восстанавливается терм \newline
    car(Lastname, ford, pink, \_).
                       \\ \hline

    34
                       &
\textbf{Сравнение:} \newline
  car(igorev, tesla,   purple, 44000)
    \newline и \newline
    car(Lastname, ford, pink, \_).
    \newline

    \textbf{Результат:} \newline
    Унификация не выполнена.
                       &
    \textbf{Прямой ход.}
                       \\ \hline

    35
                       &
\textbf{Сравнение:} \newline
  car(stasov, bmw,     green,  3700)
    \newline и \newline
    car(Lastname, ford, pink, \_).
    \newline

    \textbf{Результат:} \newline
    Унификация не выполнена.
                       &
    \textbf{Прямой ход.}
                       \\ \hline

    36
                       &
\textbf{Сравнение:} \newline
  car(alkema, lexus,   yellow, 14000)
    \newline и \newline
    car(Lastname, ford, pink, \_).
    \newline

    \textbf{Результат:} \newline
    Унификация не выполнена.
                       &
    \textbf{Прямой ход.}
                       \\ \hline

    37
                       &
    Процедура исчерпана.
    \newline

    \textbf{Результат:} \newline
    Тупик, унификация провалена.
                       &
    \textbf{Откат.}
    \newline

    Из стека восстанавливается терм \newline
    search(ford, pink, Lastname, City, Phonenum, Bankname).
    \newline

    Резольвенту покидают:

    car(Lastname, Carbrand, Color, \_),

    phonebook(Lastname, Phonenum, address(City, \_, \_, \_)),

    depositor(Lastname, Bankname, \_, \_).
                       \\ \hline

    38
                       &
    Процедура исчерпана.
    \newline

    \textbf{Результат:} \newline
    Тупик, унификация провалена.
                       &
    Резольвенту покидает

    search(ford, pink, Lastname, City, Phonenum, Bankname).
    \newline

    Резольвента и стек пусты. Поиск решений завершён.

    \textbf{Найдено 2 решения.}
                       \\ \hline
\end{longtable}
\normalsize

\footnotesize
\begin{longtable}{|c|p{.4625\textwidth}|p{.3625\textwidth}|}
    \caption{Поиск решения в примере \textnumero2}\label{tbl:ex2} \\
    \hline
    \textnumero{} шага & Сравниваемые термы; результат; подстановка (если есть)  & Дальнейшие действия: прямой ход или откат \\
    \hline

    1
                       &
\textbf{Сравнение:} \newline
  search(bmw, green, Lastname, City, Phonenum, Bankname)
    \newline и \newline
    search(Carbrand, Color, Lastname, City, Phonenum, Bankname).
    \newline

    \textbf{Результат:} \newline
    Унификация выполнена.\newline{}Carbrand=bmw; Color=green.
                       &
    В стек откладывается\newline
  search(bmw, green, Lastname, City, Phonenum, Bankname).
    \newline

    \textbf{Прямой ход.}

    На резольвенту попадают:

    car(Lastname, Carbrand, Color, \_),

    phonebook(Lastname, Phonenum, address(City, \_, \_, \_)),

    depositor(Lastname, Bankname, \_, \_).
                       \\ \hline

    2
                       &
    \textbf{Сравнение:} \newline
  car(petrov, ford,    pink,   24000)
    \newline и \newline
    car(Lastname, bmw, green, \_).
    \newline

    \textbf{Результат:} \newline
    Унификация не выполнена.
                       &
    \textbf{Прямой ход.}
                       \\ \hline

    3
                       &
    \textbf{Сравнение:} \newline
  car(petrov, ferrari, red,    55000)
    \newline и \newline
    car(Lastname, bmw, green, \_).
    \newline

    \textbf{Результат:} \newline
    Унификация не выполнена.
                       &
    \textbf{Прямой ход.}
                       \\ \hline

    4
                       &
    \textbf{Сравнение:} \newline
  car(ivanov, ford,    pink,   25000)
    \newline и \newline
    car(Lastname, bmw, green, \_).
    \newline

    \textbf{Результат:} \newline
    Унификация не выполнена.
                       &
    \textbf{Прямой ход.}
                       \\ \hline

    5
                       &
    \textbf{Сравнение:} \newline
  car(igorev, tesla,   purple, 44000)
    \newline и \newline
    car(Lastname, bmw, green, \_).
    \newline

    \textbf{Результат:} \newline
    Унификация не выполнена.
                       &
    \textbf{Прямой ход.}
                       \\ \hline

    6
                       &
\textbf{Сравнение:} \newline
  car(stasov, bmw,     green,  3700)
    \newline и \newline
    car(Lastname, bmw, green, \_).
    \newline

    \textbf{Результат:} \newline
    Унификация выполнена.\newline{}
    Lastname=stasov.
                       &
    В стек откладывается\newline
    car(Lastname, bmw, green, \_).
    \newline

    \textbf{Прямой ход.}
                       \\ \hline

    7
                       &
\textbf{Сравнение:} \newline
  phonebook(petrov, 74400297,
            address(moscow,    lenina,       4,  2))
    \newline и \newline
    phonebook(stasov, Phonenum, address(City, \_, \_, \_)).
    \newline

    \textbf{Результат:} \newline
    Унификация не выполнена.
                       &
    \textbf{Прямой ход.}
                       \\ \hline

    8
                       &
\textbf{Сравнение:} \newline
  phonebook(igorev, 77270935,
            address(moscow,    marksa,       3,  5))
    \newline и \newline
    phonebook(stasov, Phonenum, address(City, \_, \_, \_)).
    \newline

    \textbf{Результат:} \newline
    Унификация не выполнена.
                       &
    \textbf{Прямой ход.}
                       \\ \hline

    9
                       &
\textbf{Сравнение:} \newline
  phonebook(ivanov, 79345669,
            address(moscow,    pushkinskaya, 11, 1))
    \newline и \newline
    phonebook(stasov, Phonenum, address(City, \_, \_, \_)).
    \newline

    \textbf{Результат:} \newline
    Унификация не выполнена.
                       &
    \textbf{Прямой ход.}
                       \\ \hline

    10
                       &
\textbf{Сравнение:} \newline
  phonebook(stasov, 74024456,
            address(spb,       marksa,       4,  4))
    \newline и \newline
    phonebook(stasov, Phonenum, address(City, \_, \_, \_)).
    \newline

    \textbf{Результат:} \newline
    Унификация выполнена.\newline{}
    Phonenum=74024456, City=spb.
                       &
    В стек откладывается\newline
    phonebook(stasov, Phonenum, address(City, \_, \_, \_)).
    \newline

    \textbf{Прямой ход.}
                       \\ \hline

    11
                       &
\textbf{Сравнение:} \newline
  depositor(petrov, agricole, 5, 52150322)
    \newline и \newline
    depositor(stasov, Bankname, \_).
    \newline

    \textbf{Результат:} \newline
    Унификация не выполнена.
                       &
    \textbf{Прямой ход.}
                       \\ \hline

    12
                       &
\textbf{Сравнение:} \newline
  depositor(igorev, paribas,  4, 32242424)
    \newline и \newline
    depositor(stasov, Bankname, \_).
    \newline

    \textbf{Результат:} \newline
    Унификация не выполнена.
                       &
    \textbf{Прямой ход.}
                       \\ \hline

    13
                       &
\textbf{Сравнение:} \newline
  depositor(ivanov, sberbank, 6, 442423123)
    \newline и \newline
    depositor(stasov, Bankname, \_).
    \newline

    \textbf{Результат:} \newline
    Унификация не выполнена.
                       &
    \textbf{Прямой ход.}
                       \\ \hline

    14
                       &
\textbf{Сравнение:} \newline
  depositor(stasov, sberbank, 1, 423424233)
    \newline и \newline
    depositor(stasov, Bankname, \_).
    \newline

    \textbf{Результат:} \newline
    Унификация выполнена.\newline{}
    Bankname=sberbank.
                       &
    \textbf{Получено первое решение:}\newline
    Lastname=sberbank, City=spb, Phonenum=74024456, Bankname=sberbank.
    \newline

    В стек откладывается\newline
    depositor(stasov, Bankname, \_).
    \newline

    \textbf{Прямой ход.}
                       \\ \hline

    15
                       &
    Резольвента пуста.
    \newline

    \textbf{Результат:} \newline
    Тупик, унификация провалена.
                       &
    \textbf{Откат.}
    \newline

    Из стека восстанавливается терм \newline
    depositor(stasov, Bankname, \_).
                       \\ \hline

    16
                       &
\textbf{Сравнение:} \newline
  depositor(igorev, paribas,  3, 41424214)
    \newline и \newline
    depositor(stasov, Bankname, \_).
    \newline

    \textbf{Результат:} \newline
    Унификация не выполнена.
                       &
    \textbf{Прямой ход.}
                       \\ \hline

    17
                       &
\textbf{Сравнение:} \newline
  depositor(igorev, sberbank, 8, 421342352)
    \newline и \newline
    depositor(stasov, Bankname, \_).
    \newline

    \textbf{Результат:} \newline
    Унификация не выполнена.
                       &
    \textbf{Прямой ход.}
                       \\ \hline

    18
                       &
    Процедура исчерпана.
    \newline

    \textbf{Результат:} \newline
    Тупик, унификация провалена.
                       &
    \textbf{Откат.}
    \newline

    Из стека восстанавливается терм \newline
    phonebook(stasov, Phonenum, address(City, \_, \_, \_)).
                       \\ \hline

    19
                       &
\textbf{Сравнение:} \newline
  phonebook(alkema, 73148253,
            address(ekb,       marksa,       6,  8))
    \newline и \newline
    phonebook(stasov, Phonenum, address(City, \_, \_, \_)).
    \newline

    \textbf{Результат:} \newline
    Унификация не выполнена.
                       &
    \textbf{Прямой ход.}
                       \\ \hline

    20
                       &
\textbf{Сравнение:} \newline
  phonebook(igorev, 73243243,
            address(volgograd, lenina,       9,  9))
    \newline и \newline
    phonebook(stasov, Phonenum, address(City, \_, \_, \_)).
    \newline

    \textbf{Результат:} \newline
    Унификация не выполнена.
                       &
    \textbf{Прямой ход.}
                       \\ \hline

    21
                       &
    Процедура исчерпана.
    \newline

    \textbf{Результат:} \newline
    Тупик, унификация провалена.
                       &
    \textbf{Откат.}
    \newline

    Из стека восстанавливается терм \newline
    car(Lastname, ford, pink, \_).
                       \\ \hline

    22
                       &
\textbf{Сравнение:} \newline
  car(alkema, lexus,   yellow, 14000)
    \newline и \newline
    car(Lastname, ford, pink, \_).
    \newline

    \textbf{Результат:} \newline
    Унификация не выполнена.
                       &
    \textbf{Прямой ход.}
                       \\ \hline

    23
                       &
    Процедура исчерпана.
    \newline

    \textbf{Результат:} \newline
    Тупик, унификация провалена.
                       &
    \textbf{Откат.}
    \newline

    Из стека восстанавливается терм \newline
  search(bmw, green, Lastname, City, Phonenum, Bankname).
    \newline

    Резольвенту покидают:

    car(Lastname, Carbrand, Color, \_),

    phonebook(Lastname, Phonenum, address(City, \_, \_, \_)),

    depositor(Lastname, Bankname, \_, \_).
                       \\ \hline

    24
                       &
    Процедура исчерпана.
    \newline

    \textbf{Результат:} \newline
    Тупик, унификация провалена.
                       &
    Резольвенту покидает

    search(bmw, green, Lastname, City, Phonenum, Bankname).
    \newline

    Резольвента и стек пусты. Поиск решений завершён.

    \textbf{Найдено 1 решение.}
                       \\ \hline
\end{longtable}
\normalsize

\footnotesize
\begin{longtable}{|c|p{.4625\textwidth}|p{.3625\textwidth}|}
    \caption{Поиск решения в примере \textnumero3}\label{tbl:ex3} \\
    \hline
    \textnumero{} шага & Сравниваемые термы; результат; подстановка (если есть)  & Дальнейшие действия: прямой ход или откат \\
    \hline

    1
                       &
\textbf{Сравнение:} \newline
  search(lexus, yellow, Lastname, City, Phonenum, Bankname)
    \newline и \newline
    search(Carbrand, Color, Lastname, City, Phonenum, Bankname).
    \newline

    \textbf{Результат:} \newline
    Унификация выполнена.\newline{}Carbrand=lexus; Color=yellow.
                       &
    В стек откладывается\newline
  search(lexus, yellow, Lastname, City, Phonenum, Bankname).
    \newline

    \textbf{Прямой ход.}

    На резольвенту попадают:

    car(Lastname, Carbrand, Color, \_),

    phonebook(Lastname, Phonenum, address(City, \_, \_, \_)),

    depositor(Lastname, Bankname, \_, \_).
                       \\ \hline

    2
                       &
    \textbf{Сравнение:} \newline
  car(petrov, ford,    pink,   24000)
    \newline и \newline
    car(Lastname, lexus, yellow, \_).
    \newline

    \textbf{Результат:} \newline
    Унификация не выполнена.
                       &
    \textbf{Прямой ход.}
                       \\ \hline

    3
                       &
    \textbf{Сравнение:} \newline
  car(petrov, ferrari, red,    55000)
    \newline и \newline
    car(Lastname, lexus, yellow, \_).
    \newline

    \textbf{Результат:} \newline
    Унификация не выполнена.
                       &
    \textbf{Прямой ход.}
                       \\ \hline

    4
                       &
    \textbf{Сравнение:} \newline
  car(ivanov, ford,    pink,   25000)
    \newline и \newline
    car(Lastname, lexus, yellow, \_).
    \newline

    \textbf{Результат:} \newline
    Унификация не выполнена.
                       &
    \textbf{Прямой ход.}
                       \\ \hline

    5
                       &
    \textbf{Сравнение:} \newline
  car(igorev, tesla,   purple, 44000)
    \newline и \newline
    car(Lastname, lexus, yellow, \_).
    \newline

    \textbf{Результат:} \newline
    Унификация не выполнена.
                       &
    \textbf{Прямой ход.}
                       \\ \hline

    6
                       &
    \textbf{Сравнение:} \newline
  car(stasov, bmw,     green,  3700)
    \newline и \newline
    car(Lastname, lexus, yellow, \_).
    \newline

    \textbf{Результат:} \newline
    Унификация не выполнена.
                       &
    \textbf{Прямой ход.}
                       \\ \hline

    7
                       &
\textbf{Сравнение:} \newline
  car(alkema, lexus,   yellow, 14000).
    \newline и \newline
    car(Lastname, lexus, yellow, \_).
    \newline

    \textbf{Результат:} \newline
    Унификация выполнена.\newline{}
    Lastname=alkema.
                       &
    В стек откладывается\newline
    car(Lastname, lexus, yellow, \_).
    \newline

    \textbf{Прямой ход.}
                       \\ \hline

    8
                       &
\textbf{Сравнение:} \newline
  phonebook(petrov, 74400297,
            address(moscow,    lenina,       4,  2))
    \newline и \newline
    phonebook(alkema, Phonenum, address(City, \_, \_, \_)).
    \newline

    \textbf{Результат:} \newline
    Унификация не выполнена.
                       &
    \textbf{Прямой ход.}
                       \\ \hline

    9
                       &
\textbf{Сравнение:} \newline
  phonebook(igorev, 77270935,
            address(moscow,    marksa,       3,  5))
    \newline и \newline
    phonebook(alkema, Phonenum, address(City, \_, \_, \_)).
    \newline

    \textbf{Результат:} \newline
    Унификация не выполнена.
                       &
    \textbf{Прямой ход.}
                       \\ \hline

    10
                       &
\textbf{Сравнение:} \newline
  phonebook(ivanov, 79345669,
            address(moscow,    pushkinskaya, 11, 1))
    \newline и \newline
    phonebook(alkema, Phonenum, address(City, \_, \_, \_)).
    \newline

    \textbf{Результат:} \newline
    Унификация не выполнена.
                       &
    \textbf{Прямой ход.}
                       \\ \hline

    11
                       &
\textbf{Сравнение:} \newline
  phonebook(stasov, 74024456,
            address(spb,       marksa,       4,  4))
    \newline и \newline
    phonebook(alkema, Phonenum, address(City, \_, \_, \_)).
    \newline

    \textbf{Результат:} \newline
    Унификация не выполнена.
                       &
    \textbf{Прямой ход.}
                       \\ \hline

    12
                       &
\textbf{Сравнение:} \newline
  phonebook(alkema, 73148253,
            address(ekb,       marksa,       6,  8))
    \newline и \newline
    phonebook(alkema, Phonenum, address(City, \_, \_, \_)).
    \newline

    \textbf{Результат:} \newline
    Унификация выполнена.\newline{}
    Phonenum=73148253, City=ekb.
                       &
    В стек откладывается\newline
    phonebook(alkema, Phonenum, address(City, \_, \_, \_)).
    \newline

    \textbf{Прямой ход.}
                       \\ \hline

    13
                       &
\textbf{Сравнение:} \newline
  depositor(petrov, agricole, 5, 52150322)
    \newline и \newline
    depositor(alkema, Bankname, \_).
    \newline

    \textbf{Результат:} \newline
    Унификация не выполнена.
                       &
    \textbf{Прямой ход.}
                       \\ \hline

    14
                       &
\textbf{Сравнение:} \newline
  depositor(igorev, paribas,  4, 32242424)
    \newline и \newline
    depositor(alkema, Bankname, \_).
    \newline

    \textbf{Результат:} \newline
    Унификация не выполнена.
                       &
    \textbf{Прямой ход.}
                       \\ \hline

    15
                       &
\textbf{Сравнение:} \newline
  depositor(ivanov, sberbank, 6, 442423123)
    \newline и \newline
    depositor(alkema, Bankname, \_).
    \newline

    \textbf{Результат:} \newline
    Унификация не выполнена.
                       &
    \textbf{Прямой ход.}
                       \\ \hline

    16
                       &
\textbf{Сравнение:} \newline
  depositor(stasov, sberbank, 1, 423424233)
    \newline и \newline
    depositor(alkema, Bankname, \_).
    \newline

    \textbf{Результат:} \newline
    Унификация не выполнена.
                       &
    \textbf{Прямой ход.}
                       \\ \hline

    17
                       &
\textbf{Сравнение:} \newline
  depositor(igorev, paribas,  3, 41424214).
    \newline и \newline
    depositor(alkema, Bankname, \_).
    \newline

    \textbf{Результат:} \newline
    Унификация не выполнена.
                       &
    \textbf{Прямой ход.}
                       \\ \hline

    18
                       &
\textbf{Сравнение:} \newline
  depositor(igorev, sberbank, 8, 421342352)
    \newline и \newline
    depositor(alkema, Bankname, \_).
    \newline

    \textbf{Результат:} \newline
    Унификация не выполнена.
                       &
    \textbf{Прямой ход.}
                       \\ \hline

    19
                       &
    Процедура исчерпана.
    \newline

    \textbf{Результат:} \newline
    Тупик, унификация провалена.
                       &
    \textbf{Откат.}
    \newline

    Из стека восстанавливается терм \newline
    phonebook(alkema, Phonenum, address(City, \_, \_, \_)).
                       \\ \hline

    20
                       &
\textbf{Сравнение:} \newline
  phonebook(igorev, 73243243,
            address(volgograd, lenina,       9,  9))
    \newline и \newline
    phonebook(alkema, Phonenum, address(City, \_, \_, \_)).
    \newline

    \textbf{Результат:} \newline
    Унификация не выполнена.
                       &
    \textbf{Прямой ход.}
                       \\ \hline

    21
                       &
    Процедура исчерпана.
    \newline

    \textbf{Результат:} \newline
    Тупик, унификация провалена.
                       &
    \textbf{Откат.}
    \newline

    Из стека восстанавливается терм \newline
    car(Lastname, lexus, yellow, \_).
                       \\ \hline

    22
                       &
    Процедура исчерпана.
    \newline

    \textbf{Результат:} \newline
    Тупик, унификация провалена.
                       &
    \textbf{Откат.}
    \newline

    Из стека восстанавливается терм \newline
  search(lexus, yellow, Lastname, City, Phonenum, Bankname).
    \newline

    Резольвенту покидают:

    car(Lastname, Carbrand, Color, \_),

    phonebook(Lastname, Phonenum, address(City, \_, \_, \_)),

    depositor(Lastname, Bankname, \_, \_).
                       \\ \hline

    23
                       &
    Процедура исчерпана.
    \newline

    \textbf{Результат:} \newline
    Тупик, унификация провалена.
                       &
    Резольвенту покидает

    search(lexus, yellow, Lastname, City, Phonenum, Bankname).
    \newline

    Резольвента и стек пусты. Поиск решений завершён.

    \textbf{Найдено 0 решений.}
                       \\ \hline
\end{longtable}
\normalsize

\renewcommand{\arraystretch}{1}

\section{Задание \textnumero2}

В программе на Prolog база знаний формируется из правил. Причём правила, определяющие одну процедуру, располагаются вместе. Prolog проводит унификацию термов, составляющих тело правила, в порядке их следования при определении. Обход базы знаний начинается с поиска подходящей процедуры и заканчивается при рассмотрении её последнего правила. Следовательно объём работ не зависит от порядка расположения процедур в базе знаний. А порядок найденных решений соответствует порядку расположенных в процедуре правил.

\section{Задание \textnumero3}
\footnotesize
\begin{longtable}{|p{.0475\textwidth}|p{.22\textwidth}|p{.3\textwidth}|p{.06\textwidth}|p{.22\textwidth}|}
    \caption{Подробное описание унификации цели и заголовка правила}\label{tbl:task3-1} \\
    \hline
    \textnumero{} шага & результирующая ячейка & рабочее поле & пункт алгоритма & стек \\
    \hline

0
                       &

                       &

                       &
1.
                       &
search(ford, pink, Lastname, City, Phonenum, Bankname)
\newline = \newline
search(Carbrand, Color, Lastname, City, Phonenum, Bankname)
                       \\ \hline

1
                       &

                       &
search(ford, pink, Lastname, City, Phonenum, Bankname)
\newline = \newline
search(Carbrand, Color, Lastname, City, Phonenum, Bankname)

\hfill\contour{black}{$\xrightarrow{\hspace{0.13\textwidth}}$}
                       &
е)
                       &
Carbrand=ford,

Color=pink,

Lastname=Lastname,

City=City,

Phonenum=Phonenum,

Bankname=Bankname
                       \\ \hline

2
                       &
Carbrand=ford
                       &
Carbrand=ford

\contour{black}{$\xleftarrow{\hspace{0.13\textwidth}}$}
                       &
г)
                       &
Color=pink,

Lastname=Lastname,

City=City,

Phonenum=Phonenum,

Bankname=Bankname
                       \\ \hline

3
                       &
Carbrand=ford,

Color=pink
                       &
Color=pink

\contour{black}{$\xleftarrow{\hspace{0.13\textwidth}}$}
                       &
г)
                       &
Lastname=Lastname,

City=City,

Phonenum=Phonenum,

Bankname=Bankname
                       \\ \hline

4
                       &
Carbrand=ford,

Color=pink

Lastname=Lastname
                       &
Lastname=Lastname

\contour{black}{$\xleftarrow{\hspace{0.13\textwidth}}$}
                       &
г)
                       &
City=City,

Phonenum=Phonenum,

Bankname=Bankname
                       \\ \hline

5
                       &
Carbrand=ford,

Color=pink

Lastname=Lastname,

City=City
                       &
City=City

\contour{black}{$\xleftarrow{\hspace{0.13\textwidth}}$}
                       &
г)
                       &
Phonenum=Phonenum,

Bankname=Bankname
                       \\ \hline

6
                       &
Carbrand=ford,

Color=pink

Lastname=Lastname,

City=City,

Phonenum=Phonenum
                       &
Phonenum=Phonenum

\contour{black}{$\xleftarrow{\hspace{0.13\textwidth}}$}
                       &
г)
                       &
Bankname=Bankname
                       \\ \hline

7
                       &
Carbrand=ford,

Color=pink

Lastname=Lastname,

City=City,

Phonenum=Phonenum,

Bankname=Bankname
                       &
Bankname=Bankname

\contour{black}{$\xleftarrow{\hspace{0.13\textwidth}}$}
                       &
г)
                       &
                       \\ \hline

                       &
\textbf{подстановка}
                       &
\multicolumn{3}{c|}{успех~--- переход к подцели}
                       \\ \hline
\end{longtable}
\normalsize

\footnotesize
\begin{longtable}{|p{.0475\textwidth}|p{.22\textwidth}|p{.3\textwidth}|p{.06\textwidth}|p{.22\textwidth}|}
    \caption{Подробное описание унификации цели и правила}\label{tbl:task3-2} \\
    \hline
    \textnumero{} шага & результирующая ячейка & рабочее поле & пункт алгоритма & стек \\
    \hline

0
                       &

                       &

                       &
1.
                       &
search(bmw, green, Lastname, City, Phonenum, Bankname)
\newline = \newline
search(Carbrand, Color, Lastname, City, Phonenum, Bankname)
                       \\ \hline

1
                       &

                       &
search(bmw, green, Lastname, City, Phonenum, Bankname)
\newline = \newline
search(Carbrand, Color, Lastname, City, Phonenum, Bankname)

\hfill\contour{black}{$\xrightarrow{\hspace{0.13\textwidth}}$}
                       &
е)
                       &
Carbrand=bmw,

Color=green,

Lastname=Lastname,

City=City,

Phonenum=Phonenum,

Bankname=Bankname
                       \\ \hline

2
                       &
Carbrand=bmw
                       &
Carbrand=bmw

\contour{black}{$\xleftarrow{\hspace{0.13\textwidth}}$}
                       &
г)
                       &
Color=green,

Lastname=Lastname,

City=City,

Phonenum=Phonenum,

Bankname=Bankname
                       \\ \hline

3
                       &
Carbrand=bmw,

Color=green
                       &
Color=green

\contour{black}{$\xleftarrow{\hspace{0.13\textwidth}}$}
                       &
г)
                       &
Lastname=Lastname,

City=City,

Phonenum=Phonenum,

Bankname=Bankname
                       \\ \hline

4
                       &
Carbrand=bmw,

Color=green,

Lastname=Lastname
                       &
Lastname=Lastname

\contour{black}{$\xleftarrow{\hspace{0.13\textwidth}}$}
                       &
г)
                       &
City=City,

Phonenum=Phonenum,

Bankname=Bankname
                       \\ \hline

5
                       &
Carbrand=bmw,

Color=green,

Lastname=Lastname,

City=City
                       &
City=City

\contour{black}{$\xleftarrow{\hspace{0.13\textwidth}}$}
                       &
г)
                       &
Phonenum=Phonenum,

Bankname=Bankname
                       \\ \hline

6
                       &
Carbrand=bmw,

Color=green,

Lastname=Lastname,

City=City,

Phonenum=Phonenum
                       &
Phonenum=Phonenum

\contour{black}{$\xleftarrow{\hspace{0.13\textwidth}}$}
                       &
г)
                       &
Bankname=Bankname
                       \\ \hline

7
                       &
Carbrand=bmw,

Color=green,

Lastname=Lastname,

City=City,

Phonenum=Phonenum,

Bankname=Bankname
                       &
Bankname=Bankname

\contour{black}{$\xleftarrow{\hspace{0.13\textwidth}}$}
                       &
г)
                       &
                       \\ \hline

                       &
\textbf{подстановка}
                       &
                       \multicolumn{3}{p{0.58\textwidth}|}{успех~--- переход к подцели\newline{}car(Lastname, Carbrand, Color, \_)}
                       \\ \hline

8
                       &
                       &
                       &
1.
                       &
car(Lastname, bmw, green, \_)
\newline = \newline
car(petrov, ford,    pink,   24000)
                       \\ \hline

9
                       &
                       &
car(Lastname, bmw, green, \_)
\newline = \newline
car(petrov, ford,    pink,   24000)

\hfill\contour{black}{$\xrightarrow{\hspace{0.13\textwidth}}$}
                       &
е)
                       &
Lastname=petrov,

bmw=ford,

green=pink
                       \\ \hline

10
                       &
Lastname=petrov
                       &
Lastname=petrov

\contour{black}{$\xleftarrow{\hspace{0.13\textwidth}}$}
                       &
г)
                       &
bmw=ford,

green=pink
                       \\ \hline

11
                       &
Lastname=petrov
                       &
bmw=ford
                       &
г)
                       &
green=pink
                       \\ \hline

                       &
---
                       &
\multicolumn{3}{c|}{неудача~--- подбор следующего факта}
                       \\ \hline

\ldots
                       &
---
                       &
\multicolumn{3}{c|}{пропуск невыполненных унификаций}
                       \\ \hline

23
                       &
                       &
                       &
1.
                       &
car(Lastname, bmw, green, \_)
\newline = \newline
car(stasov, bmw,     green,  3700)
                       \\ \hline

24
                       &
                       &
car(Lastname, bmw, green, \_)
\newline = \newline
car(stasov, bmw,     green,  3700)

\hfill\contour{black}{$\xrightarrow{\hspace{0.13\textwidth}}$}
                       &
е)
                       &
Lastname=stasov,

bmw=bmw,

green=green
                       \\ \hline

25
                       &
Lastname=stasov
                       &
Lastname=stasov

\contour{black}{$\xleftarrow{\hspace{0.13\textwidth}}$}
                       &
е)
                       &
bmw=bmw,

green=green
                       \\ \hline

26
                       &
Lastname=stasov
                       &
bmw=bmw
                       &
б)
                       &
green=green
                       \\ \hline

27
                       &
Lastname=stasov
                       &
green=green
                       &
б)
                       &
                       \\ \hline

                       &
\textbf{подстановка}
                       &
\multicolumn{3}{p{.58\textwidth}|}{успех~--- переход к подцели\newline{}phonebook(Lastname, Phonenum, address(City, \_, \_, \_))}
                       \\ \hline

\ldots
                       &
---
                       &
\multicolumn{3}{c|}{пропуск невыполненных унификаций}
                       \\ \hline

36
                       &
                       &
                       &
1.
                       &
phonebook(stasov, Phonenum, address(City, \_, \_, \_))
\newline = \newline
phonebook(stasov, 74024456, address(spb,       marksa,       4,  4)).
                       \\ \hline

37
                       &
                       &
phonebook(stasov, Phonenum, address(City, \_, \_, \_))
\newline = \newline
phonebook(stasov, 74024456, address(spb,       marksa,       4,  4)).

\hfill\contour{black}{$\xrightarrow{\hspace{0.13\textwidth}}$}
                       &
е)
                       &
stasov=stasov,

Phonenum=74024456,

address(City, \_, \_, \_)\newline=\newline{}address(spb,       marksa,       4,  4)
                       \\ \hline

38
                       &
                       &
stasov=stasov
                       &
б)
                       &
Phonenum=74024456,

address(City, \_, \_, \_)\newline=\newline{}address(spb,       marksa,       4,  4)
                       \\ \hline

39
                       &
Phonenum=74024456
                       &
Phonenum=74024456

\contour{black}{$\xleftarrow{\hspace{0.13\textwidth}}$}
                       &
г)
                       &
address(City, \_, \_, \_)\newline=\newline{}address(spb,       marksa,       4,  4)
                       \\ \hline

40
                       &
Phonenum=74024456
                       &
address(City, \_, \_, \_)\newline=\newline{}address(spb,       marksa,       4,  4)

\hfill\contour{black}{$\xrightarrow{\hspace{0.13\textwidth}}$}
                       &
е)
                       &
City=spb
                       \\ \hline

41
                       &
Phonenum=74024456

City=spb
                       &
City=spb

\contour{black}{$\xleftarrow{\hspace{0.13\textwidth}}$}
                       &
г)
                       &
                       \\ \hline

                       &
\textbf{подстановка}
                       &
\multicolumn{3}{p{.58\textwidth}|}{успех~--- переход к подцели\newline{}depositor(Lastname, Bankname, \_, \_)}
                       \\ \hline

\ldots
                       &
---
                       &
\multicolumn{3}{c|}{пропуск невыполненных унификаций}
                       \\ \hline

50
                       &
                       &
                       &
1.
                       &
depositor(stasov, Bankname, \_, \_)
\newline = \newline
depositor(stasov, sberbank, 1, 423424233)
                       \\ \hline

51
                       &
                       &
depositor(stasov, Bankname, \_, \_)
\newline = \newline
depositor(stasov, sberbank, 1, 423424233)

\hfill\contour{black}{$\xrightarrow{\hspace{0.13\textwidth}}$}
                       &
е)
                       &
stasov=stasov,

Bankname=sberbank
                       \\ \hline

52
                       &
                       &
stasov=stasov

\contour{black}{$\xleftarrow{\hspace{0.13\textwidth}}$}
                       &
б)
                       &
Bankname=sberbank
                       \\ \hline

53
                       &
Bankname=sberbank
                       &
Bankname=sberbank

\contour{black}{$\xleftarrow{\hspace{0.13\textwidth}}$}
                       &
г)
                       &
                       \\ \hline

                       &
\textbf{подстановка}
                       &
\multicolumn{3}{p{.58\textwidth}|}{успех~--- возврат к\newline{}phonebook(Lastname, Phonenum, address(City, \_, \_, \_))}
                       \\ \hline

\ldots
                       &
---
                       &
\multicolumn{3}{c|}{пропуск оставшихся сравнений}
                       \\ \hline

\end{longtable}
\normalsize

