\chapter{Теоретическая часть}

В этом разделе приведены ответы на контрольные вопросы.

\section{В какой части правила сформулировано знание? Это знание о чем, с формальной точки зрения?}
Знание сформулировано в заголовке правила. Оно означает, что между аргументами этого правила существует определённое отношение.

\section{Что такое процедура?}
Процедурой называют совокупность правил, описывающих одно определённое отношение.

\section{Сколько в БЗ текущего задания процедур?}
4 процедуры.

\section{Что такое пример терма, это частный случай терма, пример? Как строится пример?}

Пусть $A, B$~--- термы. Терм $B$ называют примером терма $A$, если для $A$ существует такая подстановка $\alpha$, что $A\alpha= B$, где $A\alpha$~--- это результат применения подстановки $\alpha$ к терму $A$.

На мой взгляд, примеры термов строятся при поиске решения заданной пользователем цели или внутренних целей, а хранятся они до получения решения.

\section{Что такое наиболее общий пример?}
Пусть $A$, $B$~--- термы. Тогда некоторый терм $C$ называют общим примером $A$ и $B$, если $\exists$ подстановки $\alpha$, $\beta$: $A\alpha = C$, $B\beta = C$.

\section{Назначение и результат работы алгоритма унификации. Что значит двунаправленная передача параметров при работе алгоритма унификации, поясните на примере одного из случаев пункта 3.}
Алгоритм унификации предназначен для формализации процесса логического вывода. При сопоставлении двух термов пытается построить для них общий пример, это нужно для того, чтобы формально определить подходящее, при поиске ответа на вопрос, знание. При помощи алгоритма унификации происходит двунаправленная передача параметров процедурам.

\section{В каком случае запускается механизм отката?}
Откат дает возможность получить много решений в одном вопросе к программе.

Во всех точках программы, где существуют альтернативы, в стек заносятся точки возврата.

Если впоследствии окажется, что выбранный вариант не приводит к успеху, то осуществляется откат к последней из имеющихся в стеке точек программы, где был выбран один из альтернативных вариантов.

Выбирается очередной вариант, программа продолжает свою работу. Если все варианты в точке уже были использованы, то регистрируется неудачное завершение и осуществляется переход на предыдущую точку возврата, если такая есть.

При откате все связанные переменные, которые были означены после этой точки, опять освобождаются.

\section{Виды и назначение переменных в Prolog. Примеры из задания. Почему использованы те или другие переменные (примеры из задания)?}
Переменные являются частью процесса сопоставления и предназначены для передачи значений, но не для хранения их. Виды переменных:
\begin{itemize}
    \item именованная~--- обозначается комбинацией символов латинского алфавита, цифр и символа подчеркивания, начинающейся с прописной буквы или символа подчеркивания (<<X>>, <<A21>>, <<\_X>>);
    \item анонимная~--- обозначается символом подчеркивания <<\_>>. Любая анонимная переменная уникальна.
\end{itemize}

Во время вычисления, именованные переменные могут конкретизироваться. Кроме того, они могут быть конкретизированы повторно путем <<отката>> вычислительного процесса и отмены ранее проведенной конкретизации для нахождения новых решений.

Анонимные переменные не могут быть связаны со значениями.

В листинге~\ref{lst:exvar1} приведены примеры использования переменных.
\begin{lstlisting}[caption={Примеры переменных},label=lst:exvar1]
  search(_, red, Lastname, _, Phonenum, Bankname).
\end{lstlisting}
Здесь именованные переменные используются для получения набора решений описанной цели, а анонимные~--- для игнорирования некоторых значений.

