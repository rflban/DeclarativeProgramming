\chapter{Теоретические сведения}
В этом разделе будут приведены ответы на контрольные вопросы.

\section{Представление программы на Prolog}

Prolog - декларативный язык программирования. Программа на нём является набором фактов и правил, обеспечивающих получение заключений на основе этих утверждений. Prolog включает в себя механизм вывода, основанный на сопоставлении образцов термов, например терма вопроса и терма факта. Программа на Prolog представляет собой базу знаний и вопрос.

\section{Структура программы на Prolog}

Программа на Prolog состоит из следующих разделов:
\begin{enumerate}
    \item директивы компилятора - зарезервированные символьные константы;
    \item \textbf{constants} - раздел описания констант;
    \item \textbf{domains} - раздел описания доменов;
    \item \textbf{database} - раздел описания предикатов внутренней базы данных;
    \item \textbf{predicates} - раздел описания предикатов;
    \item \textbf{clauses} - раздел описания утверждений базы знаний;
    \item \textbf{goal} - раздел описания внутренней цели (вопроса).
\end{enumerate}

\section{Формирование результата работы программы на Prolog}

С помощью подбора ответов на запросы Prolog извлекает известную в программе информацию. База знаний содержит истинностные знания, используя которые программа выдает ответ на запрос. Одной из особенностей Prolog является то, что при поиске ответов на вопрос, он рассматривает альтернативные варианты и находит все возможные решения - множества значений переменных, при которых на поставленный вопрос можно ответить ``да''.

