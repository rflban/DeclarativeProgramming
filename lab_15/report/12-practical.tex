\chapter{Практическая часть}
В этом разделе приведено описание задания и его решения.

\section{Условие}
Создать базу знаний «Собственники», дополнив базу знаний, хранящую знания (лаб. 13):
\begin{itemize}
    \item \textbf{<<Телефонный справочник>>}: Фамилия, \textnumero{}тел, Адрес~--- структура (Город, Улица, \textnumero{}дома, \textnumero{}кв);
    \item \textbf{<<Автомобили>>}:  Фамилия\_владельца, Марка, Цвет, Стоимость, и др.;
    \item \textbf{<<Вкладчики банков>>}: Фамилия, Банк, счет, сумма, др.
\end{itemize}
знаниями о дополнительной \textbf{собственности} владельца. \textbf{Преобразовать} знания об \underline{автомобиле} к форме знаний о собственности.

Вид собственности (кроме автомобиля):
\begin{itemize}
    \item \textbf{Строение, стоимость} и другие его характеристики;
        \item \textbf{Участок, стоимость} и другие его характеристики;
        \item \textbf{Водный транспорт, стоимость} и другие его характеристики.
\end{itemize}

Описать и использовать вариантный домен: Собственность. Владелец может иметь, но только один объект каждого вида собственности (это касается и автомобиля), или не иметь некоторых видов собственности.

Используя конъюнктивное правило и разные формы задания одного вопроса (пояснять для какого \textnumero{}задания~--- какой вопрос), обеспечить возможность поиска:
\begin{enumerate}
    \item Названий всех объектов собственности заданного субъекта,
    \item Названий и стоимости всех объектов собственности заданного субъекта,
    \item Разработать правило, позволяющее найти суммарную стоимость всех объектов собственности заданного субъекта.
\end{enumerate}

Для 2-го пункта и одной фамилии составить таблицу, отражающую конкретный порядок работы системы, с объяснениями порядка работы и особенностей использования доменов.

\section{База знаний}

\begin{lstlisting}[caption={База знаний},label={lst:knowledge}]
domains
  lastname, city, street, carbrand, color, bankname, shipname = symbol.
  phonenum, housenum, apartnum, price, account, deposit = integer.
  area=real.

  address = address(city, street, housenum, apartnum).
  own = car(price, carbrand, color);
        building(price, address);
        landlot(price, area);
        ship(price, shipname).
predicates
  phonebook(lastname, phonenum, address).
  depositor(lastname, bankname, account, deposit).
  owner(lastname, own).

  owns(lastname, symbol, price).
clauses
  phonebook(petrov, 74400297,
            address(moscow,    lenina,       4,  2)).
  phonebook(igorev, 77270935,
            address(moscow,    marksa,       3,  5)).
  phonebook(ivanov, 79345669,
            address(moscow,    pushkinskaya, 11, 1)).
  phonebook(stasov, 74024456,
            address(spb,       marksa,       4,  4)).
  phonebook(igorev, 73243243,
            address(volgograd, lenina,       9,  9)).

  depositor(petrov, agricole, 5, 52150322).
  depositor(igorev, sberbank, 8, 421342352).
  depositor(ivanov, sberbank, 6, 442423123).
  depositor(igorev, paribas,  3, 41424214).

  owner(petrov, car     (24000,    ford,    pink)).
  owner(stasov, car     (55000,    tesla,    purple)).
  owner(ivanov, building(429999,   address(ekb, lenina, 2,  5))).
  owner(petrov, landlot (23449999, 43000)).
  owner(igorev, ship    (86000,    chehov)).

  owns(Lastname, Category, Price) :-
    owner(Lastname, car(Price, \_, \_)), Category = car.
  owns(Lastname, Category, Price) :-
    owner(Lastname, building(Price, \_)), Category = building.
  owns(Lastname, Category, Price) :-
    owner(Lastname, landlot(Price, \_)), Category = landlot.
  owns(Lastname, Category, Price) :-
    owner(Lastname, ship(Price, \_)), Category = ship.
\end{lstlisting}

\section{Цели}
Рассмотрим примеры трёх целей, текст и решение которых приведены в листингах~\ref{lst:ex1},~\ref{lst:ex2}.

\begin{lstlisting}[caption={Пример \textnumero1},label={lst:ex1}]
goal
  owns(petrov, Category, _).

% Category=car
% Category=landlot
% 2 Solutions
\end{lstlisting}

\begin{lstlisting}[caption={Пример \textnumero2},label={lst:ex2}]
goal
  owns(stasov, Category, Price).

% Category=car, Price=55000
% 1 Solution
\end{lstlisting}

\renewcommand{\arraystretch}{1.75}
\footnotesize
\begin{longtable}{|c|p{.4625\textwidth}|p{.3625\textwidth}|}
    \caption{Поиск решения в примере \textnumero1}\label{tbl:ex2} \\
    \hline
    \textnumero{} шага & Сравниваемые термы; результат; подстановка (если есть)  & Дальнейшие действия: прямой ход или откат \\
    \hline

1
                       &
\textbf{Сравнение:}\newline
  owns(stasov, Category, Price)
  \newline{}и\newline
  owns(Lastname, Category, Price)
  \newline

  \textbf{Результат:}\newline
  Унификация выполнена.

  Lastname=stasov,

  Category=Category,

  Price=Price
                       &
  В стек откладывается терм

  owns(stasov, Category, Price).
  \newline

  \textbf{Прямой ход.}
  \newline

  На резольвенту попадают:

    owner(Lastname, car(Price, \_, \_)),

    Category = car.
                       \\ \hline

2
                       &
\textbf{Сравнение:}\newline
  owner(stasov, car(Price, \_, \_))
  \newline{}и\newline
  owner(petrov, car     (24000,    ford,    pink))
  \newline

  \textbf{Результат:}\newline
  Унификация не выполнена.
                       &
  \textbf{Прямой ход.}
                       \\ \hline

3
                       &
\textbf{Сравнение:}\newline
  owner(stasov, car(Price, \_, \_))
  \newline{}и\newline
  owner(stasov, car     (55000,    tesla,    purple))
  \newline

  \textbf{Результат:}\newline
  Унификация выполнена.

  Price=5500
                       &
  В стек откладывается терм

  owner(stasov, car(Price, \_, \_))
  \newline

  \textbf{Прямой ход.}
                       \\ \hline

4
                       &
  Category = car~--- терм, в котором используется встроенный предикат =, проводящий попытку унификации передаваемых аргументов.
  \newline

  \textbf{Результат:}\newline
  Унификация выполнена.

  Category=car
                       &
  \textbf{Получено первое решение:}\newline
  Category=car, Price=55000.
  \newline

  В стек откладывается терм

  Category = car
  \newline

  \textbf{Прямой ход.}
                       \\ \hline

5
                       &
  Резольвента пуста.

  \textbf{Результат:}\newline
  Унификация провалена.
                       &
  Из стека восстанавливается терм

  Category = car
  \newline

  \textbf{Откат.}
                       \\ \hline

6
                       &
  Процедура исчерапана.

  \textbf{Результат:}\newline
  Унификация провалена.
                       &
  Из стека восстанавливается терм

  owner(stasov, car(Price, \_, \_))
  \newline

  \textbf{Откат.}
                       \\ \hline

7
                       &
\textbf{Сравнение:}\newline
  owner(stasov, car(Price, \_, \_))
  \newline{}и\newline
  owner(ivanov, building(429999,   address(ekb, lenina, 2,  5)))
  \newline

  \textbf{Результат:}\newline
  Унификация не выполнена.
                       &
  \textbf{Прямой ход.}
                       \\ \hline

8
                       &
\textbf{Сравнение:}\newline
  owner(stasov, car(Price, \_, \_))
  \newline{}и\newline
  owner(petrov, landlot (23449999, 43000))
  \newline

  \textbf{Результат:}\newline
  Унификация не выполнена.
                       &
  \textbf{Прямой ход.}
                       \\ \hline

9
                       &
\textbf{Сравнение:}\newline
  owner(stasov, car(Price, \_, \_))
  \newline{}и\newline
  owner(igorev, ship    (86000,    chehov)).
  \newline

  \textbf{Результат:}\newline
  Унификация не выполнена.
                       &
  \textbf{Прямой ход.}
                       \\ \hline

10
                       &
  Процедура исчерпана.

  \textbf{Результат:}\newline
  Унификация провалена.
                       &
  Из стека восстанавливается терм

  owner(stasov, car(Price, \_, \_))
  \newline

  \textbf{Откат.}
  \newline

  Стек покидают термы:

    owner(Lastname, car(Price, \_, \_)),

    Category = car.
                       \\ \hline

11
                       &
\textbf{Сравнение:}\newline
  owns(stasov, Category, Price)
  \newline{}и\newline
  owns(Lastname, Category, Price)
  \newline

  \textbf{Результат:}\newline
  Унификация выполнена.

  Lastname=stasov,

  Category=Category,

  Price=Price
                       &
  В стек откладывается терм

  owns(stasov, Category, Price).
  \newline

  \textbf{Прямой ход.}
  \newline

  На резольвенту попадают:

    owner(Lastname, building(Price, \_)),

    Category = building.
                       \\ \hline

12
                       &
\textbf{Сравнение:}\newline
  owner(stasov, building(Price, \_))
  \newline{}и\newline
  owner(petrov, car     (24000,    ford,    pink)).
  \newline

  \textbf{Результат:}\newline
  Унификация не выполнена.
                       &
  \textbf{Прямой ход.}
                       \\ \hline

13
                       &
\textbf{Сравнение:}\newline
  owner(stasov, building(Price, \_))
  \newline{}и\newline
  owner(stasov, car     (55000,    tesla,    purple)).
  \newline

  \textbf{Результат:}\newline
  Унификация не выполнена.
                       &
  \textbf{Прямой ход.}
                       \\ \hline

14
                       &
\textbf{Сравнение:}\newline
  owner(stasov, building(Price, \_))
  \newline{}и\newline
  owner(ivanov, building(429999,   address(ekb, lenina, 2,  5))).
  \newline

  \textbf{Результат:}\newline
  Унификация не выполнена.
                       &
  \textbf{Прямой ход.}
                       \\ \hline

15
                       &
\textbf{Сравнение:}\newline
  owner(stasov, building(Price, \_))
  \newline{}и\newline
  owner(petrov, landlot (23449999, 43000)).
  \newline

  \textbf{Результат:}\newline
  Унификация не выполнена.
                       &
  \textbf{Прямой ход.}
                       \\ \hline

16
                       &
\textbf{Сравнение:}\newline
  owner(stasov, building(Price, \_))
  \newline{}и\newline
  owner(igorev, ship    (86000,    chehov)).
  \newline

  \textbf{Результат:}\newline
  Унификация не выполнена.
                       &
  \textbf{Прямой ход.}
                       \\ \hline

17
                       &
  Процедура исчерпана.

  \textbf{Результат:}\newline
  Унификация провалена.
                       &
  Из стека восстанавливается терм

  owner(stasov, building(Price, \_))
  \newline

  \textbf{Откат.}
  \newline

  Стек покидают термы:

    owner(stasov, building(Price, \_)),

    Category = building.
                       \\ \hline

18
                       &
\textbf{Сравнение:}\newline
  owns(stasov, Category, Price)
  \newline{}и\newline
  owns(Lastname, Category, Price)
  \newline

  \textbf{Результат:}\newline
  Унификация выполнена.

  Lastname=stasov,

  Category=Category,

  Price=Price
                       &
  В стек откладывается терм

  owns(stasov, Category, Price).
  \newline

  \textbf{Прямой ход.}
  \newline

  На резольвенту попадают:

    owner(Lastname, landlot(Price, \_)),

    Category = landlot.
                       \\ \hline

19
                       &
\textbf{Сравнение:}\newline
  owner(stasov, landlot(Price, \_))
  \newline{}и\newline
  owner(petrov, car     (24000,    ford,    pink)).
  \newline

  \textbf{Результат:}\newline
  Унификация не выполнена.
                       &
  \textbf{Прямой ход.}
                       \\ \hline

20
                       &
\textbf{Сравнение:}\newline
  owner(stasov, landlot(Price, \_))
  \newline{}и\newline
  owner(stasov, car     (55000,    tesla,    purple)).
  \newline

  \textbf{Результат:}\newline
  Унификация не выполнена.
                       &
  \textbf{Прямой ход.}
                       \\ \hline

21
                       &
\textbf{Сравнение:}\newline
  owner(stasov, landlot(Price, \_))
  \newline{}и\newline
  owner(ivanov, building(429999,   address(ekb, lenina, 2,  5))).
  \newline

  \textbf{Результат:}\newline
  Унификация не выполнена.
                       &
  \textbf{Прямой ход.}
                       \\ \hline

22
                       &
\textbf{Сравнение:}\newline
  owner(stasov, landlot(Price, \_))
  \newline{}и\newline
  owner(petrov, landlot (23449999, 43000)).
  \newline

  \textbf{Результат:}\newline
  Унификация не выполнена.
                       &
  \textbf{Прямой ход.}
                       \\ \hline

23
                       &
\textbf{Сравнение:}\newline
  owner(stasov, landlot(Price, \_))
  \newline{}и\newline
  owner(igorev, ship    (86000,    chehov)).
  \newline

  \textbf{Результат:}\newline
  Унификация не выполнена.
                       &
  \textbf{Прямой ход.}
                       \\ \hline

24
                       &
  Процедура исчерпана.

  \textbf{Результат:}\newline
  Унификация провалена.
                       &
  Из стека восстанавливается терм

  owns(stasov, Category, Price)
  \newline

  \textbf{Откат.}
  \newline

  Стек покидают термы:

    owner(stasov, landlot(Price, \_)),

    Category = landlot.
                       \\ \hline

25
                       &
\textbf{Сравнение:}\newline
  owns(stasov, Category, Price)
  \newline{}и\newline
  owns(Lastname, Category, Price)
  \newline

  \textbf{Результат:}\newline
  Унификация выполнена.

  Lastname=stasov,

  Category=Category,

  Price=Price
                       &
  В стек откладывается терм

  owns(stasov, Category, Price).
  \newline

  \textbf{Прямой ход.}
  \newline

  На резольвенту попадают:

    owner(Lastname, ship(Price, \_)),

    Category = ship.
                       \\ \hline

26
                       &
\textbf{Сравнение:}\newline
  owner(stasov, ship(Price, \_))
  \newline{}и\newline
  owner(petrov, car     (24000,    ford,    pink)).
  \newline

  \textbf{Результат:}\newline
  Унификация не выполнена.
                       &
  \textbf{Прямой ход.}
                       \\ \hline

27
                       &
\textbf{Сравнение:}\newline
  owner(stasov, ship(Price, \_))
  \newline{}и\newline
  owner(stasov, car     (55000,    tesla,    purple)).
  \newline

  \textbf{Результат:}\newline
  Унификация не выполнена.
                       &
  \textbf{Прямой ход.}
                       \\ \hline

28
                       &
\textbf{Сравнение:}\newline
  owner(stasov, ship(Price, \_))
  \newline{}и\newline
  owner(ivanov, building(429999,   address(ekb, lenina, 2,  5))).
  \newline

  \textbf{Результат:}\newline
  Унификация не выполнена.
                       &
  \textbf{Прямой ход.}
                       \\ \hline

29
                       &
\textbf{Сравнение:}\newline
  owner(stasov, ship(Price, \_))
  \newline{}и\newline
  owner(petrov, landlot (23449999, 43000)).
  \newline

  \textbf{Результат:}\newline
  Унификация не выполнена.
                       &
  \textbf{Прямой ход.}
                       \\ \hline

30
                       &
\textbf{Сравнение:}\newline
  owner(stasov, ship(Price, \_))
  \newline{}и\newline
  owner(igorev, ship    (86000,    chehov)).
  \newline

  \textbf{Результат:}\newline
  Унификация не выполнена.
                       &
  \textbf{Прямой ход.}
                       \\ \hline

31
                       &
  Процедура исчерпана.

  \textbf{Результат:}\newline
  Унификация провалена.
                       &
  Из стека восстанавливается терм
  owns(stasov, Category, Price)

  \newline

  \textbf{Откат.}
  \newline

  Стек покидают термы:

    owner(stasov, ship(Price, \_)),

    Category = ship.
                       \\ \hline

32
                       &
  Процедура исчерпана.

  \textbf{Результат:}\newline
  Унификация провалена.
                       &
  Стек покидает терм

  owns(stasov, Category, Price)
  \newline

  Поиск решений завершен. \textbf{Найдено одно решение.}
                       \\ \hline
\end{longtable}
\normalsize

