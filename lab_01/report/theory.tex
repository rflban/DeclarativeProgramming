\chapter{Теоретические сведения}
\textbf{Базис} в Lisp образуют атомы, структуры, базовые функции, базовые функционалы.

Вся информация в Lisp представляется в виде S-выражений. \textbf{S-выражения} - это атом или точечная пара.
\textbf{Атомами} являются:
\begin{itemize}
    \item \textbf{символы} - идентификаторы, набор букв и цифр, начинающийся с буквы;
    \item \textbf{T}, \textbf{Nil} - специальные символы, использующиеся для обозначения логических констант. Кроме того, Nil означает пустой список, а все значения, не равные Nil, интерпретируются как T;
    \item \textbf{самоопределяемые атомы} - натуральные числа, дробные числа, вещественные числа, строки.
\end{itemize}

Так же существуют более сложные структуры данных - списки и точечные пары.

\textbf{Точечной парой} является конструкция вида (A . B), где под A и B подразумевается либо атом, либо точечная пара.

\textbf{Список} быть пустым или не пустым. \textbf{Пустой список} - это () или, как было упомянуто выше, Nil. \textbf{Не пустой список} - это точечная пара, состоящая из головы и хвоста. \textbf{Голова списка} - это S-выражение, а \textbf{хвост} - список.

