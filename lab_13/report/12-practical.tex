\chapter{Практическая часть}
В этом разделе приведено описание задания и его решения.

\section{Условие}
Составить программу, т.е. модель предметной области~--- базу знаний, объединив в ней информацию~--- знания:
\begin{itemize}
    \item \textbf{<<Телефонный справочник>>}: Фамилия, \textnumero{}тел, Адрес~--- структура (Город, Улица, \textnumero{}дома, \textnumero{}кв);
    \item \textbf{<<Автомобили>>}:  Фамилия\_владельца, Марка, Цвет, Стоимость, и др.;
    \item \textbf{<<Вкладчики банков>>}: Фамилия, Банк, счет, сумма, др.
\end{itemize}
Владелец может иметь несколько телефонов, автомобилей, вкладов (Факты).

Используя правила, обеспечить возможность поиска:
\begin{enumerate}
    \item \textbf{а)} По \textnumero{} телефона найти: Фамилию, Марку автомобиля, Стоимость автомобиля (может быть несколько);

        \textbf{б)} Используя сформированное в пункте \textbf{(а)} правило, по \textnumero{} телефона найти: только Марку автомобиля (автомобилей может быть несколько);

    \item Используя простой, не составной вопрос: по Фамилии (уникальна в городе, но в разных городах есть однофамильцы) и Городу проживания найти: Улицу проживания, Банки, в которых есть вклады и \textnumero{} телефона.
\end{enumerate}

Для одного из вариантов ответов, и для (а) и для (б), описать словесно порядок поиска ответа на вопрос, указав, как выбираются знания, и, при этом, для каждого этапа унификации, выписать подстановку~--- наибольший общий унификатор, и соответствующие примеры термов.

\section{База знаний}

В листинге~\ref{lst:knowledge} приведён текст базы знаний, удовлевторяющей условию задачи.
\begin{lstlisting}[caption={База знаний},label={lst:knowledge}]
domains
  lastname, city, street, carbrand, color, bankname = symbol.
  phonenum, housenum, apartnum, price, account, deposit = integer.
  address = address(city, street, housenum, apartnum).
predicates
  phonebook(lastname, phonenum, address).
  car(lastname, carbrand, color, price).
  depositor(lastname, bankname, account, deposit).

  search(phonenum, lastname, carbrand, price).
  search(phonenum, carbrand).
  search(lastname, city, street, bankname, phonenum).
clauses
  phonebook(petrov, 74400297,
            address(moscow,    lenina,       4,  2)).
  phonebook(igorev, 77270935,
            address(moscow,    marksa,       3,  5)).
  phonebook(ivanov, 79345669,
            address(moscow,    pushkinskaya, 11, 1)).
  phonebook(ivanov, 70244559,
            address(spb,       lenina,       3,  5)).
  phonebook(stasov, 74024456,
            address(spb,       marksa,       4,  4)).
  phonebook(petrov, 78771841,
            address(ekb,       lenina,       7,  9)).
  phonebook(igorev, 73148253,
            address(ekb,       marksa,       6,  8)).
  phonebook(igorev, 73243243,
            address(volgograd, lenina,       9,  9)).

  car(petrov, bugatti, red,    2400000).
  car(petrov, ferrari, red,    55000).
  car(ivanov, ford,    pink,   25000).
  car(igorev, tesla,   purple, 44000).
  car(igorev, bmw,     green,  3700).
  car(igorev, lexus,   yellow, 14000).

  depositor(petrov, agricole, 5, 52150322).
  depositor(igorev, paribas,  4, 32242424).
  depositor(ivanov, sberbank, 2, 242342142).
  depositor(ivanov, sberbank, 6, 442423123).
  depositor(stasov, sberbank, 1, 423424233).
  depositor(petrov, agricole, 6, 42421234).
  depositor(igorev, paribas,  3, 41424214).
  depositor(igorev, sberbank, 8, 421342352).

  search(Phonenum, Lastname, Carbrand, Price) :-
    phonebook(Lastname, Phonenum, _),
    car(Lastname, Carbrand, _, Price).

  search(Phonenum, Carbrand) :-
    search(Phonenum, _, Carbrand, _).

  search(Lastname, City, Street, Bankname, Phonenum) :-
    phonebook(Lastname, Phonenum, address(City, Street, _, _)),
    depositor(Lastname, Bankname, _, _).
\end{lstlisting}

\section{Цели}
Рассмотрим примеры целей для каждого задания.

\subsection{Задание 1.а}
В листингах~\ref{lst:aex1},~\ref{lst:aex2},~\ref{lst:aex3} приведены цели, использующие предикат search(phonenum, lastname, carbrand, price), а так же их решение.

\begin{lstlisting}[caption={Пример \textnumero1},label={lst:aex1}]
goal
  search(73243243, Lastname, Carbrand, Price).

% Lastname=igorev, Carbrand=tesla, Price=44000
% Lastname=igorev, Carbrand=bmw, Price=3700
% Lastname=igorev, Carbrand=lexus, Price=14000
% 3 Solutions
\end{lstlisting}

\begin{lstlisting}[caption={Пример \textnumero2},label={lst:aex2}]
goal
  search(79345669, Lastname, Carbrand, Price).

% Lastname=ivanov, Carbrand=ford, Price=25000
% 1 Solution
\end{lstlisting}

\begin{lstlisting}[caption={Пример \textnumero3},label={lst:aex3}]
goal
  search(09345669, Lastname, Carbrand, Price).

% No Solution
\end{lstlisting}

Рассмотрим подробнее поиск ответа на вопрос из листинга~\ref{lst:aex2}. Описание этого процесса приведено в таблице~\ref{tbl:ex1}.
\renewcommand{\arraystretch}{1.75}
\footnotesize
\begin{longtable}{|c|p{.4625\textwidth}|p{.3625\textwidth}|}
    \caption{Поиск решения в примере \textnumero2}\label{tbl:ex1} \\
    \hline
    \textnumero{} шага & Сравниваемые термы; результат; подстановка (если есть)  & Дальнейшие действия: прямой ход или откат \\
    \hline
        1 & search(79345669, Lastname, Carbrand, Price) и search(Phonenum, Lastname, Carbrand, Price) совпадают по имени и списку аргументов $\Rightarrow$ происходит унификация:
        \begin{enumerate}[label=\arabic*)]
			\item переменная Phonenum унифицируется с константой 79345669 , т.е. конкретизируются её значением;
			\item переменная Lastname унифицируется с переменной Lastname, т.е. остаётся неконкретизированной;
			\item Carbrand унифицируется с Carbrand;
			\item Price унифицируется с Price.
 		\end{enumerate}
        Результат~--- search(79345669, Lastname, Carbrand, Price). &
        В стек откладывается терм search(79345669, Lastname, Carbrand, Price).

        Прямой ход: на резольвенту попадают термы, из которых состоит тело правила search(Phonenum, Lastname, Carbrand, Price).

        На вершине стека вопросов находится подцель phonebook(Lastname, Phonenum, \_).
        \\ \hline
        2 & phonebook(Lastname, Phonenum, \_) и phonebook(petrov, 74400297, address(moscow,    lenina,       4,  2)) унифицируются:
        \begin{enumerate}[label=\arabic*)]
            \item Lastname конкретизируется значением petrov;
            \item конкретизированная значением 79345669 переменная Phonenum не унифицируются с константой 74400297, так как их значения не совпадают.
        \end{enumerate}
        Унификация не выполняется. & Прямой ход. \\ \hline

        3 & phonebook(Lastname, Phonenum, \_) и phonebook(igorev, 77270935, address(moscow,    marksa,       3,  5)) не унифицируются, потому что значение конкретизированной переменной Phonenum и константы 77270935 не совпадают. & Прямой ход. \\ \hline

        4 & phonebook(Lastname, Phonenum, \_) и phonebook(ivanov, 79345669, address(moscow,    pushkinskaya, 11, 1)) унифицируются:
        \begin{enumerate}[label=\arabic*)]
            \item Lastname конкретизируется значением константы ivanov;
            \item конкретизированная переменная Phonenum унифицируется с константой 79345669, так как их значения совпадают;
            \item анонимная переменная унифицируется с константой address(\dots).
        \end{enumerate}
        Результат~--- phonebook(igorev, 79345669, \_, 3,  5)). &
        В стек откладывается терм  phonebook(Lastname, Phonenum, \_).

        Прямой ход: на вершине стека вопросов находится вторая подцель~--- car(Lastname, Carbrand, \_, Price). \\ \hline

        5 & car(Lastname, Carbrand, \_, Price) и car(petrov, bugatti, red,    2400000) унифицируются:
        \begin{enumerate}[label=\arabic*)]
            \item конкретизированная значением ivanov переменная Lastname не унифицируется с константой petrov.
        \end{enumerate}
          Унификация не выполняется. &
          Прямой ход. \\ \hline

        6 & car(Lastname, Carbrand, \_, Price) и car(petrov, ferrari, red,    55000) унифицируются:
        \begin{enumerate}[label=\arabic*)]
            \item конкретизированная значением ivanov переменная Lastname не унифицируется с константой petrov.
        \end{enumerate}
          Унификация не выполняется. &
          Прямой ход. \\ \hline

        7 & car(Lastname, Carbrand, \_, Price) и car(ivanov, ford, pink, 25000) унифицируются:
        \begin{enumerate}[label=\arabic*)]
            \item конкретизированная значением ivanov переменная Lastname унифицируется с константой ivanov.
            \item Carbrand конкретизируется значением константы ford;
            \item анонимная переменная унифицируется с константой pink;
            \item Price конкретизируется значением константы 25000.
        \end{enumerate}
        Результат~--- car(ivanov, ford, \_, 25000). &

        \textbf{Найдено первое решение: Lastname = ivanov; Carbrand = ford; Price = 25000.}

        В стек откладывается терм car(Lastname, Carbrand, \_, Price). Резольвента опустела.

        Откат: отложенный терм восстанавливается. \\ \hline

        8 & car(Lastname, Carbrand, \_, Price) не унифицируется с car(igorev, tesla,   purple, 44000) & Прямой ход. \\ \hline

        9 & car(Lastname, Carbrand, \_, Price) не унифицируется с car(igorev, bmw,   green, 3700) & Прямой ход. \\ \hline

        10 & car(Lastname, Carbrand, \_, Price) не унифицируется с car(igorev, lexus,   yellow, 14000) & Обход базы знаний завершен.

        Откат: восстановление из стека терма phonebook(Lastname, Phonenum, \_). \\ \hline

        11--15 & phonebook(Lastname, Phonenum, \_) больше не унифицируется ни с одним из оставшихся в базе знаний фактом или правилом. & Обход базы знаний завершен.

        Откат: восстановление из стека цели search(79345669, Lastname, Carbrand, Price) и удаление из резольвенты её подцелей. \\ \hline

        16 & search(79345669, Lastname, Carbrand, Price) больше не унифицируется ни с одним из оставшихся в базе знаний фактом или правилом. & Обход базы знаний завершен. Цель search(79345669, Lastname, Carbrand, Price) покидает резольвенту.

        Завершение поиска решений. Найдено 1 решение. \\ \hline
\end{longtable}
\normalsize

\subsection{Задание 1.б}
В листингах~\ref{lst:bex1},~\ref{lst:bex2},~\ref{lst:bex3} приведены цели, использующие предикат search(phonenum, carbrand), а так же их решение.

\begin{lstlisting}[caption={Пример \textnumero1},label={lst:bex1}]
goal
  search(73243243, Carbrand).

% Carbrand=tesla
% Carbrand=bmw
% Carbrand=lexus
% 3 Solutions
\end{lstlisting}

\begin{lstlisting}[caption={Пример \textnumero2},label={lst:bex2}]
goal
  search(79345669, Carbrand).

% Carbrand=ford
% 1 Solution
\end{lstlisting}

\begin{lstlisting}[caption={Пример \textnumero3},label={lst:bex3}]
goal
  search(09345669, Carbrand).

% No Solution
\end{lstlisting}

Рассмотрим подробнее поиск ответа на вопрос из листинга~\ref{lst:bex2}. Описание этого процесса приведено в таблице~\ref{tbl:ex2}.
\renewcommand{\arraystretch}{1.75}
\begin{table}[H]
    \centering
    \footnotesize
    \begin{longtable}{|c|m{.4625\textwidth}|m{.3625\textwidth}|}
    \caption{Поиск решения в примере \textnumero}\label{tbl:ex2} \\
    \hline
    \textnumero{} шага &
        Сравниваемые термы; результат; подстановка (если есть) &
        Дальнейшие действия: прямой ход или откат \\
    \hline
    1 &
\begin{enumerate}[label=\arabic*)]
    \item
\end{enumerate} &
         \\ \hline
\end{longtable}
\end{table}

\subsection{Задание 2}
В листингах~\ref{lst:2ex1},~\ref{lst:2ex2},~\ref{lst:2ex3} приведены цели, использующие предикат search(lastname, city, street, bankname, phonenum), а так же их решение.

\begin{lstlisting}[caption={Пример \textnumero1},label={lst:2ex1}]
goal
  search(petrov, moscow, Street, Bankname, Phonenum).

% Street=lenina, Bankname=agricole, Phonenum=74400297
% Street=lenina, Bankname=agricole, Phonenum=74400297
% 2 Solutions
\end{lstlisting}

\begin{lstlisting}[caption={Пример \textnumero2},label={lst:2ex2}]
goal
  search(stasov, spb, Street, Bankname, Phonenum).

% Street=marksa, Bankname=sberbank, Phonenum=74024456
% 1 Solution
\end{lstlisting}

\begin{lstlisting}[caption={Пример \textnumero3},label={lst:2ex3}]
goal
  search(stasov, moscow, Street, Bankname, Phonenum).

% No Solution
\end{lstlisting}

Рассмотрим подробнее поиск ответа на вопрос из листинга~\ref{lst:2ex2}. Описание этого процесса приведено в таблице~\ref{tbl:ex3}.
\begin{table}[H]
    \centering
    \footnotesize
    \begin{longtable}{|c|m{.4625\textwidth}|m{.3625\textwidth}|}
    \caption{Поиск решения в примере \textnumero}\label{tbl:ex3} \\
    \hline
    \textnumero{} шага &
        Сравниваемые термы; результат; подстановка (если есть) &
        Дальнейшие действия: прямой ход или откат \\
    \hline
    1 &
\begin{enumerate}[label=\arabic*)]
    \item
\end{enumerate} &
         \\ \hline
\end{longtable}
\end{table}

\renewcommand{\arraystretch}{1}

