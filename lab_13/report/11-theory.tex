\chapter{Теоретическая часть}

В этом разделе приведены ответы на контрольные вопросы.

\section{Что такое терм?}
Терм~--- основной элемент языка. Терм~--- это константа (число, символьный атом, строка), переменная (именованная, анонимная), составной терм.

\section{Что такое предикат в матлогике?}
Предикат~--- это логическая функция от одного или нескольких аргументов. Другими словами, предикат~--- это функция, отображающая множество произвольной природы в множество \{ложь, истина\}.

\section{Что описывает предикат в Prolog?}
Название отношений, существующих между объектами.

\section{Назовите виды предложений в программе и приведите примеры таких предложений из Вашей программы. Какие предложения являются основными, а какие~--- не основными? Каковы: синтаксис и семантика (формальный смысл) этих предложений (основных и неосновных)?}
Предложения бывают двух видов~--- факты и правила.
\begin{enumerate}
    \item Правило состоит из тела и головы. Голову так же называют заголовком. Синтаксически правило оформляется следующим образом: \textbf{голова :- тело.}

    Причем, заголовок и тело~--- это термы, а символ <<:->> это специальный символ-разделитель.
    \item Факт~--- частный случай правила, в котором отсутствует тело символ-разделитель.
\end{enumerate}

Пример факта из программы приведён в листинге~\ref{lst:exfact}.
\begin{lstlisting}[caption={Пример факта},label=lst:exfact]
phonebook(petrov, 74400297, address(moscow, lenina, 4, 2)).
\end{lstlisting}

Пример правила из программы приведён в листинге~\ref{lst:exrule}.
\begin{lstlisting}[caption={Пример правила},label=lst:exrule]
search(Phonenum, Carbrand) :- search(Phonenum, _, Carbrand, _).
\end{lstlisting}

Если составные термы, факты, правила и вопросы не содержат переменных, то они называются основными. Составные термы, факты, правила и вопросы в момент фиксации в программе могут содержать переменные, тогда они называются неосновными

\section{Каковы назначение, виды и особенности использования переменных в программе на Prolog? Какое предложение БЗ сформулировано в более общей~--- абстрактной форме: содержащее или не содержащее переменных?}
Переменные являются частью процесса сопоставления и предназначены для передачи значений, но не для хранения их. Виды переменных:
\begin{itemize}
    \item именованная~--- обозначается комбинацией символов латинского алфавита, цифр и символа подчеркивания, начинающейся с прописной буквы или символа подчеркивания (<<X>>, <<A21>>, <<\_X>>);
    \item анонимная~--- обозначается символом подчеркивания <<\_>>. Любая анонимная переменная уникальна.
\end{itemize}

Во время вычисления, именованные переменные могут конкретизироваться. Кроме того, они могут быть конкретизированы повторно путем <<отката>> вычислительного процесса и отмены ранее проведенной конкретизации для нахождения новых решений.

Анонимные переменные не могут быть связаны со значениями.

Предложения базы знаний, содержащие переменные, сформулировано в в более общей~--- абстрактной форме.

\section{Что такое подстановка?}
Пусть $A(X_1, X_2, \dots, X_n)$~--- терм. Тогда подстановкой называют множество пар вида $\{X_i = t_i\},\ i = \overline{1,n}$, где $X_i$~--- переменная, а $t_i$~--- терм.

\section{Что такое пример терма? Как и когда строится? Как Вы думаете, система строит и хранит примеры?}

Пусть $A, B$~--- термы. Терм $B$ называют примером терма $A$, если для $A$ существует такая подстановка $\alpha$, что $A\alpha= B$, где $A\alpha$~--- это результат применения подстановки $\alpha$ к терму $A$.

На мой взгляд, примеры термов строятся при поиске решения заданной пользователем цели или внутренних целей, а хранятся они до получения решения.

