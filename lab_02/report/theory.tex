\chapter{Теоретические сведения}
\textbf{Базис} в Lisp образуют атомы, структуры, базовые функции, базовые функционалы.

Классификация \textbf{функций}:
\begin{itemize}
    \item \textbf{классические функции} - математические функции, возвращающие один результат;
    \item \textbf{формы} - функции, принимающие произвольное количество аргументов, которые могут по-разному обрабатываться;
    \item \textbf{функционалы} - функции, принимающие на вход описание другой функции.
\end{itemize}

Так же существует классификация \textbf{базовых функций}:
\begin{itemize}
    \item \textbf{конструкторы} - cons, list и т.д.;
    \item \textbf{селекторы} - функции доступа (например, car, cdr);
    \item \textbf{функции-предикаты} - atom, null и т.д.;
    \item \textbf{функции сравнения} - eq, eql, = и т.д.
\end{itemize}

